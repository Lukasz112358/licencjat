\documentclass{article}
\usepackage{listings}
\usepackage[T1]{fontenc}
\usepackage{algorithm}
\usepackage{algorithmic}
\usepackage{amsfonts}
\usepackage{tikz}
\usepackage{systeme}
\usepackage{tcolorbox}
% Language setting
% Replace `english' with e.g. `spanish' to change the document language
\usepackage[polish]{babel}

% Set page size and margins
% Replace `letterpaper' with `a4paper' for UK/EU standard size
\usepackage[letterpaper,top=2cm,bottom=2cm,left=3cm,right=3cm,marginparwidth=1.75cm]{geometry}

% Useful packages
\usepackage{amsmath}
\usepackage{inconsolata}
\usepackage{graphicx}
\graphicspath{ {./} }
\usepackage[colorlinks=true, allcolors=blue]{hyperref}

\title{Algorytm znajdowania sumy podzbioru}
\author{You}

\begin{document}
\maketitle

\begin{abstract}
Problem istnienia podzbioru o sumie $t$ zawierającego się w $n$-elementowym (multi)zbiorze $S \subset \mathbb{N}$, 
jest jednym z klasycznych problemów algorytmicznych. Problem w ogólności jest problemem $NP$-zupełnym
jednak jeżeli $t$ jest względnie małe i dysponujemy pamięcią $\Omega(t)$ istnieją algorytmy działające 
w czasie wielomianowym od długości wejścia, z czego najbardziej znanym jest działający w czasie $O(nt)$
algorytm oparty na programowaniu dynamicznym.

Celem nieniejszej pracy jest dokumentacja implementacji
algorytmu opracowanego przez Ce Jin i Hongxun Wu w pracy [odnośnik!!!]. Algorytm ten jest niedeterministyczny,
jednak dla dużych danych prawdopodobieństwo błędu jest niewielkie. Złożoność czasowa prezentowanego
algorytmu to $O((t+n)\log^2(t))$, zaś pamięciowa $O(n+t)$.

\end{abstract}

\section{Wstęp}
Problem sprawdzania czy z $n-$elementowego muiltizbioru $S$ liczb naturalnych jesteśmy w stanie wybrać 
podzbiór, którego suma elementów jest równa zadanej liczbie $t$ jest jednym z klasycznych problemów 
algorytmicznych. Będziemy go w dalszej części nazywać problem sumy podzbiorów. 

Ma on zastosowanie w licznych praktycznych zagadnieniach na przykład kryptografii
[Merkle–Hellman knapsack cryptosystem], analizie finansowej 
[Solving Subset Sum Problems using Binary Optimization with 
Applications in Auditing and Financial Data Analysis], czy zagadnieniach kombinatorycznych, jako specyficzny
wariant problemu plecakowego(Problem plecakowy dotyczy wybrania ze zbirou przedmiotów, w którym każdy ma 
określoną wagę i wartość, podzbioru mającego maksymalną sumaryczną wartość i jednocześnie, którego 
sumaryczna waga nie przekracza pewnej liczby $t$. Problem sumy podzbiorów odpowiada sytuacji, 
gdy wartości przedmiotów są wprostproporcjonalne do ich wagi). 

Klasyczna wersja problem, gdzie $t$ jest duże(gdy nie dysponujemy $O(t)$ pamięci) jest problemem 
$NP$-zupełnym. Najprostszy algorytm rozwiązujący ten problem polega na rozważeniu po kolei wszystkich 
możliwych podzbiorów. Złożoność tego algorytmu to $O(n2^n)$. Nieco szybszym algorytmem jest algorytm Horowitza
i Sahaniego który działa w $O(2^{\frac{n}{2}}n/2)$, jednak w przeciwieństwie do algorytmu naiwnego wymagającego
$O(n)$ pamięci algorytm ten wymaga $O(2^{\frac{n}{2}})$. Algorytm Schroeppela i Shamira wymaga takiego samego
czasu jak algorytm Horowitza i Sahaniego, jednak potrzebuje $O(2^{\frac{n}{4}})$ pamięci. Probabilistyczny 
algorytm Howgrave-Grahama i Jouxa pozwolił przyspieszyć rozwiązanie problemu do $O(2^{0.337n})$
zużywając $O(2^{0.256})$ pamięci. Dalsze jego ulepszenia pozwoliły osiągnąć złożoność czasową równą 
$O(2^{0.291n})$.

Znacznie mniej czasu wymagają instancje problemu sumy podzbiorów gdzie $t$ jest względnie małe i dysponujemy
$O(t)$ pamięci. Oparty o programowanie dynamiczne klasyczny algorytm wymaga $O(nt)$ czasu i $O(n+t)$ pamięci.
Używam go do sprawdzenia poprawności implementacji algorytmu Ce Jin i Hongxun Wu. Najszybszym znanym algorytmem 
dla problemu sumy podbiorów z małym $t$ jest algorytm Konstantinos Koiliaris i Chao Xu. 

Algorytm który prezentuję w nieniejszej pracy został opracowany przez Ce Jin i Honguxun Wu. Działa on w czasie 
$O(n+t\log^2(t))$ i wymaga $O(n+t)$ pamięci. Jest to algorytm probabilistyczny z prawdopodobieństwem błędu $O(\frac{1}{n+t})$.
Moja implementacja składa się z 4 modułów: klasy field odpowiadającej symulacji 
działań w ciele $Z_p$ reszt z dzielenia przez liczbę pierwszą $p$, moduł losujący liczbę $p$ kożystający z testu Millera-Rabina,
moduł zawierający implementację teorio-liczbowej szybkiej transformaty Fouriera(jest to pewna nieortodoksja względem pracy
Ce Jin i Honguxun Wu, ponieważ proponowali oni szybką transformatę Fouriera, jednak początkowe próby jej implementacji
powodowały problemy związane z niedokładnością operacji zmiennoprzecinkowych) oraz implementacja algorytmu właściwego.

Moja implementacja jest napisana w języku \texttt{C++} i kożysta ze zmiennych całkowitych 128-bitowych więc kompilacja może nie 
powieść się na niektórych systemach i kompilatorach w szczególności na systemach 32-bitowych. Dokładne wymagania opiszę
w dalszej części pracy.

W nieniejszej pracy zaprezentuję niedeterministyczny algorytm opracowany przez Ce Jin 
i Hongxun Wu, który kożystając ze sprytnych obserwacji na polu analizy matematycznej i algebry jest w 
stanie podać wynik w czasie $O(n+t\log^2(t))$, co jest czasem znacznie szybszym niż klasyczny algorytm 
oparty na programowanie dynamiczne. 



\section{Algorytm klasyczny}
Specyfikacja problemu $SubsetSum$, który będziemy rozwiązywać jest następująca:

\begin{tcolorbox}
    \textbf{Wejście:} Liczba naturalna $n$ , liczba naturalna $t$, zbiór $S$ reprezentowany jako ciąg $n$ liczb naturalnych(po wczytaniu 
    przechowywanych w wektoerze \texttt{S}).
    
    \textbf{Wyjście:} \texttt{true} jeżeli istnieje $S' \subseteq S$ którego suma elementów jest równa $t$ lub \texttt{false} w przeciwnym przypadku.
\end{tcolorbox}
Lub alternatywnie
\begin{tcolorbox}
    \textbf{Wejście:} Liczba naturalna $n$ , liczba naturalna $t$, zbiór $S$ reprezentowany jako ciąg $n$ liczb naturalnych.
    
    \textbf{Wyjście:} Wektor $t+1$-elementowy wektor \texttt{ans} wartości boolowskich, taki, że dla każdego
    $i \in \{0,1,2,...,t\}$ \texttt{ans[i]=true} wtedy i tylko wtedy, gdy
    istnieje $S' \subseteq S$ którego suma elementów jest równa $t$.
\end{tcolorbox}


Klasyczny algorytym opiera się na modyfikacji $t+1$-elementowej tablicy \texttt{DP} o komórkach przyjmujących wartości bool-owskie i 
opiera się na programowaniu dynamicznym. Początkowo wszystkie komórki \texttt{DP} są ustawione na \texttt{false}, oprócz komórki 
$0$-owej, która
jest ustawiona na \texttt{true}.

Dokładny algorytm przedstawia poniższy pseudokod(odpowiada on drugiej specyfikacji problemu sumy podzbiorów. Jeżeli chcielibyśmy
odpowiedź na 1 jego wersję wystarczy zwrócić \texttt{DP[t]}, przy czym można ją zwrócić wtedy od razu gdy przyjmie wartość \texttt{true}).
\begin{lstlisting}
    zainicjuj n-elementowe DP i oldDP
    for i <-1,2,...,t+1
        oldDP[i] = false
    end for 
    DPold[0] = true
    for i<-0,1,...,n-1
        for j <- S[i],S[i]+1,...,t
            DP[j] = oldDP[j] or oldDP[j-S[i]]
        end for
        oldDP = DP
    end for
    return DP
\end{lstlisting}

Dowód poprawności tego algorytmu jest prostym dowodem indukcyjnym, w którym teza indukcyjna
brzmi: po wykonaniu $i$-tej iteracji pętli z iteratorem \texttt{i}
 \texttt{DP[m]=true}
wtedy i tylko wtedy
gdy można wybrać podzbiór zbioru
$\{S\left[0\right],S\left[1\right],..,S\left[i\right]\}$
taki, że suma jego elemetów wynosi 

 $m$(oczywiście dla $k \in \{0,1,...,t\}$). 
\begin{itemize}
    \item Dla $i = 0$ teza jest oczywista.
    \item Załóżmy, że teza jest prawdziwa dla $i-1$. Jeżeli istnieje podzbiór
    zbioru $\{S \left[ 0 \right],S\left[ 1 \right],..,S\left[ i \right]\}$ taki, że suma jego elementów wynosi $k$ to jest to 
    albo podzbiór zbioru $\{{S\left[0\right],S\left[1\right],..,S\left[i-1\right]\}}$ i z tezy indukcyjnej $DP[k]=ture$
    jeszcze przed wykonaniem $i$-tej iteracji, albo jest to podzbiór zawierający element
    $S\left[i \right]$, którego pozostałe elementy należące do 
    $\{S \left[ 0 \right] ,S \left[ 1 \right],..,S \left[ i-1 \right] \}$
     sumują się 
    do \texttt{k-S[i]}, więc z tezy indukcyjnej \texttt{DP[k-S[i]]= true} . Ponieważ \texttt{DP[k]} po wykonaniu $i$-tej iteracji
    przyjmuje wartość \texttt{DP[i] or DP[k-S[i]]} teza indukcyjna jest prawdziwa.

\end{itemize}
Ponieważ komórki \texttt{DP} przyjmują tylko wartości \texttt{true} i \texttt{false} do reprezentacji jej najoptymalniej
użyć bitsetu a kolejne iteracje pętli zewnętrznej wykonać poleceniem \texttt{DP |= (DP >> S[i])}.

Pętla zewnętrzna wykona się $O(n)$ razy i każda iteracja zajmie $O(t)$ czasu tak więc
czas działania całego algorytmu zajmie $O(nt)$ i $O(t+n)$ pamięci.



\section{Idea algorytmu Ce Jin i Hongxun Wu}


Algorytm Ce Jin i Hongxun Wu bazuje na dość prostej obserwacji, że dla danego zbioru
$S=\{s_1,s_2,...,s_n\} \subset \mathbb{N}$ istnieje jego podzbiór sumujący się do $t \in \mathbb{N}$ wtedy 
i tylko wtedy gdy, współczynnik przy $x^t$ w  wielomianie $A(x) :=\prod_{i = 1}^{n}(1+x^{s_i})$ jest 
niezerowy. Istotnie jeżeli ten współczynnik jest niezerowy, to istnieje ciąg indeksów $0<i_1<i_2<...<i_k<n+1$ taki, że 

$$\prod_{j=1}^{k}x^{s_{i_j}}=x^{\sum_{j=1}^ks_{i_j}}=x^t$$, więc ciąg ten odpowiada indeksom elementów które należy wybrać
by uzyskaźć podzbiór zbioru $S$ o sumie elementów równej $t$.
Zamiast liczyć ten współczynnik wprost najpierw obliczymy $B(x):=\ln(\prod_{i = 1}^{n}(1+x^{s_i}))$, 
zaś następnie $\exp(B(x)) = A(x)$. 

Co to jednak znaczy, że obliczymy te funkcje? Będziemy wyliczać rozwinięcia jej w szereg Taylora.
Współczynnik przy $x^t$ w rozwinięciu w szereg Taylora $\exp(B(x))$ istotnie jest współczynnikiem
przy $x^t$ w $A(x)$. Wynika to bezpośrednio z jednoznaczności rozwinięcia w szereg potęgowy i tego, że $A(x)$ jest wielomianem. 

W dalszej części sekcji będę stosował schemat notacji $F_t(x)$ jako oznaczenie rozwinięcia w szereg Taylora 
$t$-pierwszych wyrazów funkcji $F$, tak więc $\exp_t(x) = \sum_{i=0}^t\frac{x^i}{i!}$, zaś 
$\ln_t(1+x^a)=\sum_{i=0}^t\frac{(-1)^{i-1}x^{ai}}{n}$ 

Aby znaleźć odpowiedź na omawiany w tej pracy problem wystarczy oczywiście ustalić jedynie 
wartość $t$ pierwszych wyrazów rozwinięcia w szereg Taylora funkcji $\exp(B(x))$.

Niestety obliczenia potrzebne do znalezienia tych współczynników mogą wymagać użycia bardzo dużych liczb długości $O(n)$. 
Obliczenia na nich mogą być więc czasochłonne. 

Ce Jin i Hongxun Wu skożystali z faktu, że
pochodna nie jest jedynie obiektem, który można zdefiniować analitycznie jako przekształcenie funkcji 
$f(x)$ w funkcję zwracającą dla argumentu $x$ wartość $\lim_{h \to 0}\frac{f(x+h)-f(x)}{h}$, ale 
i przekształcenie czysto algebraiczne, które przekształca szereg $\sum_{i=0}^{\infty}f_i x^i$ w
$\sum_{i=1}^{\infty}if_{i}ix^{i-1}$. 

Tak zdefiniowana pochodna zachowuje pewnealgebraiczne właściwości($f'+g'=(f+g)'$,$(fg)'=f'g+g'f$,
$(f(g))'=f'(g)g'$ ) bez względu na to do jakiego ciała należą współczynniki tych szeregów, tak więc również pojęcie szeregu
Taylora z pewnymi ograniczeniami   możemy rozpatrywać w innych ciałach niż $\mathbb{R}$. W naszym przypadku będzie to ciało 
$Z_p$ reszt z dzielenia przez $p$, gdzie $p$ jest losową liczbą pierwszą na tyle dużą, że prawdopodobieństwo fałszywego zakwalifikowania
jakiejś liczby jako $0$ jest niewielkie, jednak na tyle małą, że obliczenia na elementach $Z_p$ wykonują się względnie szybko.

%Okazuje się, że w naszym algorytmie wystarczy rozpatrywać jedynie rozpatrywać rozwinięcia używanych 
%funkcji jedynie do $t$-tego wyrazu i jeżeli weźmiemy liczbę pierwszą $p>t$ to w tym przypadku
%oznacza to brak konieczności dzielenia przez liczby podzielne przez $p$, tak więc ciałem, w 
%którym możemy rozpatrywać współczynniki naszych szeregów jest ciało reszt modulo $p$ dla jakiejś losowo wybranej $p>t$. Dodawanie liczb $a$ i $b$ w takim ciele przypomina dodawanie w liczbach całkowitych jednak zamiast zwracać "cały" wynik dodawania w liczbach całkowitych
%zwracamy tylko jego resztę z dzielenia przez $p$. Mnożenie i odemowanie wykonuje się 
%analogicznie, z kolei dzielenie przez liczbę $d$ polega na pomnożeniu przez odwrotność
%$d$ w ciele $Z_p$.

%W naszym algorytmie co prawda udało się uniknąć dzielenia, przez liczby podzielne przez $p$,
%jednak wciąż jest możliwe uzyskanie liczb podzielnych przez $p$ na drodze dodawania i mnożenia, co może sprawić, że błędnie zinterpretujemy niektóre liczby jako $0$. W szczególności możliwe jest błędne zidentyfikowanie jako $0$ $t$-tego współczynnika wielomianu 
%$A(x)$ co w bezpośredni sposób może wpłynąć na poprawność zwracanej odpowiedzi. Okazuje się
%jednak, że prawdopodobieństwo takiego błędu w algorytmie Ce Jin i Honguxun Wu jest $O(n+t)^{-1}$. Ja użyłem w swojej implementacji nieco innej metody losowania tej liczby, jednak jak
%eksperymentalnie sprawdziłem można zastosować dla tego sposobu analogiczny sposób szacowania prawdopodobieństwa błędu, jaki zastosowali Ce Jin i Honguxun Wu o czym więcej będę pisał w dalszej części pracy.

Ważną optymalizacją w liczeniu współczynników rozwinięcia $\exp(B(x))$ było zastąpienie jednej podwójnej pętli, 
której wykonanie w sposób naiwny zajmowałoby pesymistycznie $O(t^2)$ czasu pomnożeniem dwóch wielomianów stopnia co 
najwyżej $O(t)$ współczynnikach, co można wykonać szybką 
transformatą Furiera w czasie $O(\log(t)t)$. Aby uniknąć problemów z utratą dokładności w obliczeniach na liczbach 
zmiennoprzecinkowych zastosowałem Teorio-liczbową szybką transformatę Furiera, w której współczynniki rozpatrywałem w ciele 
wylosowanej już na potrzeby wcześniej omawianych obliczeń 
liczby $p$.

Ogólnie implementację algorytmu można podzielić na 5 zasadniczych części
\begin{itemize}
  \item Implementacja funkcji losującą liczbę pierwszą $p$ o porządanych własnościach
  \item Implementacja klasy odpowiadającej za wykonywanie obliczeń w ciele $Z_p$
  \item Implementacja Teorio-liczbowej szybkiej transformaty Furiera w ciele $Z_p$
  \item Implementacja algorytmu właściwego
  \item Implementacja kodu testującego, w tym algorytmu naiwnego.
\end{itemize}




\section{Implementacja ciała $Z_p$}

Ogólny szablon, jakie metody należy zaimplementować był 
mocno inspirowany implementacją klasy dużych liczb całkowitych
zamieszczony na stronie [ https://www.geeksforgeeks.org/bigint-big-integers-in-c-with-example/ ]


Ponieważ znaczną część obliczeń mój program będzie wykonywać na elmentach ciała
$Z_p$ w którym istnieją działania arytmetycznych, aby uniknąć konieczności ciągłych operacji pobierania explicite
wartości modulo $p$ z wyników działań najwygodniejszym podejściem byłoby zaprogramowanie klasy \texttt{field} której instancje
reprezentują elementy ciała $Z_p$ zaś przeciążone operatory odpowiadają działaniom w $Z_p$ i pisząc bardziej 
wysokopoziomowe funkcje nie musimy już się troszczyć o to, by jakkolwiek "nromalizować" wyniki. 


Klasa ma kilka pól statycznych. Jednym z nich jest liczbna \texttt{p} \texttt{\textunderscore \textunderscore int128}
będąca liczbą pierwszą modulo którą będziemy wykonywać nasze obliczenia. Liczba \texttt{m} typu 
\texttt{\textunderscore \textunderscore int128}, która ma zastosowanie przy wykonywaniu działania mnożenia w sposób
chroniący nas przed przepełnieniem zmiennej co zostanie omówione w dalszej części. 

Liczbę \texttt{p} można zapisać jako $2^kr+1$, gdzie $r$ jest liczbą nieparzystą zaś $k$ liczbą naturalną.
W zmiennej statycznej typu \texttt{\textunderscore \textunderscore int128} i nazwie \texttt{odd} przechowujemy
wartość owego $r$, zaś w statycznej zmiennej \texttt{degreeOfDegree} typu
\texttt{\textunderscore \textunderscore int128} przechowujemy wartość owego $r$.

Dla każdej liczby pierwszej $p$ istnieje pierwiastek pierwotny, czyli taka liczba, że $r$, że nie istnieje 
dodatnia liczba naturalna $w$ mniejsza niż $p-1$ taka, że $r^w \equiv 1 (\mod p)$ (oczywiście z małego twierdzenia
Fermata $r^{p-1} \equiv 1 (\mod p)$). W dalszej części pracy stopniem liczby będę oznaczał najmniejszy wykładnik
naturalny do którego należy ją podnieść, aby uzyskać $1$ w ciele $Z_p$. Stopień pierwiastka pierwotnego wynosi 
oczywiście $p-1$. Bardzo często będziemy chcieli szybko wyliczyć liczbę mającą stopień będący potęgą $2$ o 
wykładniku naturalnym $k$. Najprościej taką liczbę uzyskać jako rezultat potęgowania pierwiastka pierwotnego,
najpierw do potęgi, której wykładnik jest równy wartości zmiennej statycznej \texttt{odd}, a następnie
wykonaniu \texttt{degreeOfDegree-k}  podniesień wyniku do kwadratu. Wygodnie byłoby nie musieć za każdym razem
wykonywać potęgowania do potęgi \texttt{odd}, dlatego w zmiennej statycznej \texttt{almostPrimitiveRoot}
typu \texttt{\textunderscore \textunderscore int128} przechowóję wartość będącą wynikiem podniesienia pierwiastka
pierwotnego do potęgi \texttt{odd}.

Ostatnim polem statycznym klasy \texttt{field} jest struktura typu 
\texttt{std::map<\textunderscore \textunderscore int128 , \textunderscore \textunderscore int128 >} i nazwie 
inverse przechowująca odwrotności elementów ciała. Będzie ona dynamicznie powiększana w trakcie wykonania 
programu i służy ominięcie konieczności wykonywania dużej ilości powtórzeń kosztownej operacji liczenia odwrotności
elementu. 



Metoda \texttt{setP} ustawia wartość
\texttt{p} na przyjętą jako argument wartość zmiennej \texttt{x} typu 
\texttt{\textunderscore \textunderscore int128}. 
Ustawia ona również zmienną \texttt{m} na wartość $\lfloor \frac{2^{127}-1}{p} \rfloor$.
Czyści tablicę \texttt{inverse}(wcześniej wyliczone odwrotności elementów
po zmianie ciała przestają być dłużej odwrotnościami tych samych elementów).
Wykonując poniższy kod znajduję wartość pól \texttt{almostPrimitiveRoot}(w zmiennej \texttt{degree}) oraz 
\texttt{degreeOfDegree}(w zmiennej \texttt{y}) i następnie przypisujemy je odpowiednim polom.
\texttt{odd}
\begin{lstlisting}
__int128 y = field::p - 1;
__int128 degree = 0;
while(y \% 2 == 0){
    degree++;
    y /= 2;
}
\end{lstlisting} 

Następnie szukamy wartości pola \texttt{almostPrimitiveRoot}. 
Przy zazłożeniu prawdziwości hipotezy Riemanna pierwiastek pierwotny modulo $p$ jest
stosunkowo mały wielkości $O(\log^6(p))$ [ŹROOOOOOOOOOOOOOODŁOOOOOOOOOOOOO!!!!!!!!!!], 
tak więc kandydatów możemy szukać sprawdzając kolejne liczby naturalne zaczynając od 2. 
Tak naprawdę nie potrzebujemy wyznaczyć pierwiastka pierwotnego, ale jedynie liczbę stopnia ale jedynie liczbę
$r$ dla której nie zachodzi $r^{2^{degreeOfDegree-1}} \equiv 1 (\mod p)$. Sprawdzenie dla liczby $r$ czy nie zachodzi 
$r^{2^\frac{p-1}{2}} \equiv 1 (\mod p)$ jest mocniejszym warunkiem i dla uproszczenia kodu on zostaje sprawdzony. Następnie
jeżeli wspomniany warunek zachodzi do zmiennej \texttt{almostPrimitiveRoot} zapisujemy $r^{odd}$.

W module zajmującym losowaniem liczb pierwszych istnieje miejsce w którym wykonujemy operacje modularne modulo liczb
które nie muszą być pierwsze jednak nie wykonujemy modulo je dzielenia. Wygodnie byłoby móc użyć metod klasy 
\texttt{field}. Aby uniknąć potencjalnych niespodziewanych zachowań programu np podczas wyznaczania wartości pola 
\texttt{almostPrimitiveRoot} podczas wywołania funkcji \texttt{setP} na 
argumencie niebędącym liczbą pierwszą utworzyłem funkcję \texttt{setPstupid}, która ustawia przyjmuje argument
\texttt{x} typu \texttt{\textunderscore \textunderscore int128} i ustawia na niego wartość
pola \texttt{p}, wartość pola \texttt{m} ustawia podobnie jak w funkcji \texttt{setP} na  
$\lfloor \frac{2^{127}-1}{p} \rfloor$, pozostałe pola statyczne zaś ustawia na \texttt{0} i czyści tablicę odwrotności.



Pole, które przechowuje wartość pojedynczego obiektu nazwałem value. Oczywiście
pole to już nie może być statyczne.
Stworzyłem kilka konstruktorów dla klasy \texttt{field}. Konstruktor domyślny
nie przyjmuje żadnego argumentu i ustawia wartość \texttt{value} na \texttt{0}.
Konstruktor kopiujący ustawia wartość \texttt{value} na wartość \texttt{value} obiektu 
typu \texttt{field} podanego jako argument. 

Ostatni konstruktor przyjmuje liczbę całkowitą \texttt{val} 
typu \texttt{\textunderscore \textunderscore int128}. Żeby uniknąć nietypowych zachowań programu chcemy, żeby wartości \texttt{value} dla 
każdego obiektu klasy \texttt{field} były mniejsze niż \texttt{p} i większe niż \texttt{0}.Jeżeli więc wartość \texttt{val}
jest większa lub równa \texttt{0}  polu \texttt{value} przypisuję
jej resztę z dzielenia przez \texttt{p}. Jeżeli jest zaś jest ujemna przypisuję mu wynik działania \texttt{((val\%p)+p)\%p}.

%przyjmują jako argument liczbę całkowitą $val$ odpowiednio typu
%$int$ i $long$ $long$ i ustawiają wartość pola $value$ na resztę z
%dzielenia $val$ przez $p$ jednak ponieważ dla uproszczenia chcemy trzymać
%jako $value$ tylko liczby nieujemne zamiast przypisać polu $value$ wprost
%wartości $val \% p$, przypisujemy jej wartość $((val\%p)+p)\%p$ co jak 
%można łatwo stwierdzić w istocie zawsze przyjmuje wartość dodatnią będącą
%resztą z dzielenia $val$ przez $p$.

Przejdę teraz do omówienia kolejnych funkcji i przeciążonych operatorów
w omawianej klasie.

Utworzyłem kilka get-terów, żeby w sposób bardziej kontrolowany odwoływać się do niepublicznych pól. 
\texttt{getP}, \texttt{getM}, \texttt{getAlmostPrimitiveRoot}, \texttt{getOdd}, \texttt{getDegreeOfDegree}
służą odpowiednio do pobrania wartości pól \texttt{p}, \texttt{m}, \texttt{almostPrimitiveRoot},

Metody \texttt{friend field inline \& operator++()}, 
\texttt{friend field inline operator++(int temp)},
\texttt{friend field inline \& operator--()},
\texttt{friend field inline \& operator++(int temp)} to odpowiednio przeciążenia
operatora postinkrementacji, preinkrementacji, postdekrementacji i predekrementacj.
Operator postinkrementacji kożystając z niezmiennika, mówiącego,
że wartość pola \texttt{value} jest zawsze liczbą całkowitą z
przedziału $[0,1,..,p-1]$ który
zachowują wszystkie funkcje modyfikujące pole \texttt{value} oraz konstruktory
może przypisać polu \texttt{value} po prostu wartość wyrażenia
\texttt{(value+1)\%field::p}. Jeżeli \texttt{value=0} naiwna postdekrementacja
powoduje uzyskanie wyniku ujemnego dlatego polu \texttt{value} w przypadku postdekrementacji
zamiast \texttt{(value+1)\%field::p} przypisujemy 
\texttt{(field::p+value-1)\%field::p}. Dla operatora 
preinkrementacji najpierw tworzymy zmienną \texttt{aux} w której przechowujemy
starą wartość obiektu, którą później zwrócimy, a następnie na oryginalnym 
obiekcie dokonujemy zdefiniowaną wcześniej postinkrementację.
Operator predekrementacji definiujemy analogicznie jak preinkrementacji.

Następnie definiuję przeciążenia operatorów 
\texttt{friend inline field \&operator+=},
\texttt{friend inline field \&operator+},
\texttt{friend inline field \&operator-=},
\texttt{friend inline field \&operator-}. Operator
\texttt{+=} przyjmuje przez referencję dwa argumenty typu \texttt{field},
gdzie pierwszy oznaczam jako \texttt{a}, zaś drugi jako \texttt{b}. Wartości pola
\texttt{value} przypisujemy wartość \texttt{( a.value + b.value ) \%field::p} gdzie 
operator \texttt{\%} służy zachowaniu niezmiennika, by wartości pola \texttt{value}
były mniejsze niż \texttt{p}. Z kolei dla operatora \texttt{-=} wartości analogicznego
pola przypisujemy wartość wyrażenia \texttt{(a.value-b.value+field::p)\%field::p}
gdzie nieobecne w poprzednim przypadku dodanie $p$ służy niedopuszczeniu
do sytuacji gdy \texttt{value} stanie się ujemne jeśli \texttt{b.value > a.value}.
W implementacji operatora \texttt{+} tworzymy tymczasowy obiekt \texttt{temp} klasy 
\texttt{field}, o wartości
początkowej pola \texttt{value} równej \texttt{a.value}, a następnie
przy pomocy zdefiniowanego wcześniej operatora \texttt{+=} dodajemy do tego obiektu
obiekt \texttt{b}. Analogicznie definiujemy operatora \texttt{-}.

Kolejnymi nieco prostszymi operatorami jakie zdefiniowałem to operatory porownójące
\texttt{friend inline bool operator==}, 
\texttt{friend inline bool operator!=}, 
\texttt{friend inline bool operator<},
\texttt{friend inline bool operator<=}, 
\texttt{friend inline bool operator>}, 
\texttt{friend inline bool operator>=} 
które dla argumentów \texttt{a} i \texttt{b} zwracają wartości boolowskie 
odpowiednio wyrażeń 
\texttt{a.value == b.value}, 
\texttt{a.value != b.value},
\texttt{a.value < b.value},
\texttt{a.value <= b.value},
\texttt{a.value >= b.value}

Nieco bardziej skomplikowana jest implementacja operatora \texttt{friend field \&operator*=}. Ponieważ liczba 
\texttt{p} mnoże być rzędu $O(t(n+t)^3)$ gdybyśmy postąpili naiwnie i (przyjmując 
oznaczenie pierwszego argumentu jako \texttt{a}, drugiego jako \texttt{b}) 
przypisali polu \texttt{value} wyniku wartość wyrażenia
\texttt{(a.value*b.value)\%field::p} oznaczałoby to pobranie reszty z dzielenia przez p w 
wartości wyrażenia \texttt{a.value*b.value}, która 
byłaby rzędu $O(t^2(n+t)^6)$, co już dla \texttt{t} rzędu $10^5$ mogłoby powodować 
wychodzenie poza zakres nawet zmiennej typu
\texttt{\textunderscore \textunderscore int128}. 
Oczywiście tak mały zakres liczb mocno wpłynąłby na użyteczność
zaimplementowanych funkcji. Definiuję więc zmienną \texttt{m}, będącą 
górną wartością \texttt{b}, dla którego możemy wykonać mnożenie w sposób naiwny 
(i następnie wyciągnąć tylko resztę z dzielenia przez \texttt{p}) bez obaw o przepełnienie.
W przeciwnym wypadku tworzę zmienną pomocniczą \texttt{A} będącą wartością pola value
obiektu \texttt{a} oraz zmienną \texttt{B} będącą wartością pola value obiektu 
\texttt{b}.
\begin{tcolorbox}
\begin{center}
    Na potrzeby dalszej części pracy definiuję operator matematyczny $\%$ który dla lewego 
    argumentu będącego liczbą naturalną $A$ i prawego będącego liczbą naturalną dodatnią 
    $B$ zwraca resztę z dzielenia $A$ przez $B$. Jest więc bardzo podobny do operatora 
    \texttt{\%} w składni C++.
\end{center}    
\end{tcolorbox}
Zauważmy, że $A \cdot B = A \cdot \lfloor \frac{B}{m} \rfloor  + A \cdot (B \% m)$
Wartość \texttt{m} jest zależna od \texttt{p} i dlatego ustawiamy ją podczas operacji 
\texttt{setP} o czym pisałem w 
jednym poprzednich akapitów. Inicjalizuję obiekt typu \texttt{field} ans1 o wartości pola \texttt{value} równej \texttt{A}.
Następnie mnożę go rekurencyjnie przy pomocy operatora \texttt{*=} przez \texttt{field(fied::m)}, a następnie przez 
\texttt{field(B/field::m)}. Inicjalizuję też obiekt \texttt{ans2} o początkowej wartości pola \texttt{value}
równej \texttt{A} i mnożę rekurencyjnie przy pomocy operatora \texttt{*=} przez \text{field(B\%field::m)}. 
Po tych operacjach wartość pola value obiektu \texttt{ans1} wynosi A $\cdot \lfloor \frac{B}{m} \rfloor$, zaś 
obiektu \texttt{ans2} $A \cdot (B \% m)$. Do \texttt{a} zapisujemy więc \texttt{ans1+ans2} i zwracamy \texttt{a}.


Kolejnym operatorem jest operator \texttt{friend field operator*}. Argumenty, które otrzymuje
to obiekty typu \texttt{field} o nazwach \texttt{a} i \texttt{b}. Tworzę tymczasowy obiekt typu \texttt{field} i nazwie \texttt{temp}
o tej samej wartości pola \texttt{value} co \texttt{a} i następnie mnożę go przez \texttt{b} przy pomocy
zdefiniowanego wcześniej operatora \texttt{*=} i zwracam tak zmodyfikowany obiekt \texttt{temp}.

Kolejnym operatorem jest operator potęgowania \texttt{friend field inline \&operator}\verb!^!\texttt{=}. 
Pierwszym argumentem jaki przyjmuje jest obiekt typu \texttt{field} o nazwie \texttt{a} oraz zmienna typu 
\texttt{\textunderscore \textunderscore int128} o
nazwie \texttt{b}. Zauważmy, że z małego twierdzenia Fermata(jeżeli $p$ nie dzieli $a$, jeżeli jednak dzieli również własność zachodzi, 
czego dowód jest trywialny) $a^{k(p-1)+l} \equiv (a^{p-1})^ka^l \equiv 1^ka^l \equiv a^l (\mod p) $ dla dowolnych naturalnych
$a,k,l$, tak więc
zamiast podnosić \texttt{b} do potęgi \texttt{b} możemy podnieść ją do potęgi \texttt{B\%(p-1)}.
Teoretycznie moglibyśmy rozpatrywać \texttt{((b\%(p-1))+p-1) \% (p - 1)}, jednak z pewnych powodów, o 
których napiszę w dalszej części pracy zdecydowałem się rozpatrzyć przypadki \texttt{b} ujemnego i 
dodatniego osobno. 

Należy również pamiętać, że po zaimplementowaniu funkcji \texttt{setPstupid}
nasza klasa może symulować niektóre działania modulo \texttt{p} dla \texttt{p}
nie będącego liczbą pierwszą. Oczywiście wtedy małe twierdzenie Fermata
przestaje pozwalać na ową redukcję \texttt{b} do reszty z dzielenia przez
\texttt{p-1}, tak więc przypadek ten skorygowałem prostym \texttt{if}-em, sprawdzającym 
czy wartość pola \texttt{odd} jest równa \texttt{0}, co może się zdarzyć tylko wtedy gdy klasa 
jest inicjalizowana funkcją \texttt{setPstupid}.

Jeżeli \texttt{b==0} przypisuję \texttt{a} wartość \texttt{1} i zwracam wynik. Jeżeli \texttt{B==1}  nie 
nie robię nic, tylko odrazu zwracam \texttt{a}. Jeżeli \texttt{B==-1} to jeżeli \texttt{a.value==0} 
wyrzucam błąd dzielenia przez \texttt{0}, jeżeli zaś nie wykonuję następującą procedurę.
Ponieważ potęgowanie do \texttt{-1} to znalezienie odwrotności \texttt{a} w $\mathbb{Z}_p$
z małego twierdzenia Fermata oznacza to podniesienie \texttt{a} do potęgi 
\texttt{p-2}. Potęgowanie można zaimplementować, by wykonywało się w czasie logarytmicznym
względem stopnia, jednak mnożenie ze względu na groźbę przepełnienia
też może w pesymistycznym przypadku zabierać logarytmicznie dużo czasu.
Znalezienie odwrotności obiektu jest dość podstawową operacją, którą 
możemy chcieć wykonywać bardzo często, tak więc czas $O(\log ^2(p))$ może
nie być zadowalający. Dlatego klasa \texttt{field} posiada jeszcze jedno pole, którym
jest statyczna mapa o nazwie \texttt{inverse} z wejściami typu 
\texttt{<\texttt{\textunderscore \textunderscore int128},\texttt{\textunderscore \textunderscore int128}>}, która przypisuje 
liczbie \texttt{x} liczbę która jest wartością odwrotną w ciele $\mathbb{Z}_p$. Mapę tą
czyścimy, za każdym razem gdy zmieniamy wartość \texttt{p}. Istotnie wtedy odwrotności
tych samych liczb mogą stać się inne, bo zmienia się ciało w którym są tymi odwrotnościami.
Operacja \texttt{insertInverse} przyjmuje zmienne typu \texttt{field}, które zakładamy
explicite, że są elementami wzajemnie odwrotnymi i oznaczamy jako \texttt{a} oraz \texttt{b}, 
a następnie, jako, że $(a^{-1})^{-1}=a$ wykonujemy od razu 2 przypisania
\begin{lstlisting}
  field::inverse[a.value] = b.value;
  field::inverse[b.value] = a.value;
\end{lstlisting}
Gdy mamy wyliczyć potęgę o wykładniku \texttt{-1} jakiegoś elementu \texttt{a} typu \texttt{field} sprawdzam 
najpierw czy w mapie odwrotności jest przypisana wartość dla klucza \texttt{a.value}, 
a następnie jeżeli kluczowi temu przypisana jest wartość ustawiamy \texttt{a.value} na nią i zwracamy \texttt{a}, zaś
w przeciwnym przypadku wykonujemy operację 
\texttt{field::insertInverse(a, A\textsuperscript{$\wedge$}=(( long long)(field::p-2)))}, gdzie \texttt{A} to kopia
obiektu \texttt{a}.
(rekurencyjnie wywołujemy operastor \texttt{\textsuperscript{$\wedge$}=} dla wartości nieujemnej)
i ponownie pobieramy wartość dla klucza \texttt{a.value}. Pozwala nam to dla każdego elementu 
$\mathbb{Z}_p$ wyliczyć
jego odwrotność co najwyżej raz, bez względu na ilość wywołań tego wyliczenia w wyżej poziomowym 
kodzie. 

Kolejnym przypadkiem, który należy rozważyć to gdy \texttt{B>1}. Wykonuję wtedy iteracyjną wersję szybkiego potęgowania.
Tworzę pomocniczy obiekt \texttt{multiplier} będący kopią \texttt{a}, oraz obiekt \texttt{ans} o
wartości pola \texttt{value} równej 1. Następnie wykonuję pętlę
\begin{lstlisting}
while(B != 0){
    if(B % 2 == 1)ans *= multiplier;
    multiplier *= multiplier;
    B /=2;
}
\end{lstlisting}
i następnie kopiuję wartość \texttt{ans} do obiektu \texttt{a} oraz zwracam \texttt{a}.

Kolejnym i już ostatnim przypadkiem jest gdy \texttt{B < -1}. W tym przypadku wykożystujemy 
fakt, że $a^B=(a^{-1})^{-B}$. Tak więc kolejno wykonujemy zdefiniowane wcześniej 
podniesienie \texttt{a} do potęgi \texttt{-1} przy pomocy operatora \texttt{\textsuperscript{$\wedge$} =} 
a następnie znów przy
pomocy tego samego operatora podnosimy \texttt{a} do potęgi \texttt{-B>1}, co też zostało już 
wcześniej zdefiniowane. Po wykonaniu tych obliczeń zwracam \texttt{a}.

Kolejnym operatorem jest \texttt{\textsuperscript{$\wedge$}}. Pobiera on argument \texttt{a} typu 
\texttt{field} i \texttt{b} typu 
\texttt{\textunderscore \textunderscore int128}, tworzy
przy pomocy konstruktóra kopiującego tymczasowy obiekt \texttt{temp} typu \texttt{field} z taką 
samą wartością pola 
\texttt{value} co \texttt{a}, następnie przy pomocy operatora 
\texttt{\textsuperscript{$\wedge$}=} podnosi 
\texttt{temp} do potęgi \texttt{B} i zwraca tak zmodyfikowaną wartość \texttt{temp}.

Kolejnym operatorem będzie operator dzielenia \texttt{\textbackslash=} przyjmujący dwa argumenty typu 
\texttt{field} oznaczone odpowiednio jako \texttt{a} i \texttt{b}. Dzielenie w ciele modulo nie oznacza
"klasycznego" dzielenia arytmetycznego, ale pomnożenie przez element odwrotny. 
Tak więc tworzymy nowy obiekt \texttt{B} będący kopią \texttt{b} i mnożymy \texttt{a} przy pomocy
operartora \texttt{*=} przez \texttt{B\textsuperscript{$\wedge$}(long$ $long)(-1)}. Następnie zwracamy tak zmodyfikowane \texttt{a}.

Kolejnym operatorem jest \texttt{\textbackslash}. Otrzymuje on dwa argumenty typu \texttt{fieled} o nazwach \texttt{a} i 
\texttt{b}. Tworzę obiekt \texttt{temp} typu field i przypisuję mu wartość \texttt{a}, następnie dzielę przez \texttt{b} przy pomocy 
operatora \texttt{\textbackslash=} i zwracam \texttt{temp}.

Kolwjnymi operatorami są operatory strumieniowe

\texttt{std::istream \&operator> > (std::istream \&in,field\&a)} i

\texttt{std::ostream \&operator< <(std::ostream \&out,const field \&a)}.

Wypisanie/wczytanie elementu będzie polegać na wypisaniu/wczytaniu 
wartości jego pola \texttt{value}. Jednak kompilator \texttt{g++} nie ma zdefiniowanych operatorów strumieniowych 
dla zmiennych typu \texttt{\textunderscore \textunderscore int128}. Dlatego będę w przypadku operatora \texttt{> >} najpierw
wczytywał zapis rządanej wartości pola \texttt{value} do zmiennej typu \texttt{std::string}, a następnie przy pomocy funkcji 
\texttt{fromString} przekształcał ją do zmiennej typu \texttt{\textunderscore \textunderscore int128} i zwracał podany 
jako argument strumień \texttt{in}. Z kolei w przypadku operatora \texttt{< <} będę konwertował zmienną typu 
\texttt{\textunderscore \textunderscore int128} do zmiennej typu \texttt{std::string} i dopiero po konwersji
wypisywał na podany jako argument strumień \texttt{out} i zwracał ten strumień.

Funkcja \texttt{fromString} używa funkcji \texttt{charToDigit}, która przyjmując znak \texttt{x} typu 
\texttt{char} będący zapisem jakiejś cyfry w języku "ludzkim" zwraca liczbę typu 
\texttt{\textunderscore \textunderscore int128} mającą wartość tej cyfry. Funkcja \texttt{fromString} przyjmuje 
zmienną \texttt{x} typu \texttt{std::string} jako argument a następnie przechodząc w pętli od tyłu
po znakach tego słowa
dodaje do zmiennej wynikowej \texttt{ans} kolejne potęgi \texttt{10} o wykładniku naturalnym pomnożone przez wartość
wyniku funkcji \texttt{charToDigit} na rozpatrywanym znaku. Jeżeli zaś napotka na znak \texttt{-} mnoży \texttt{ans}
przez \texttt{-1}.

Zmienna \texttt{fromString} kożysta z funkcji \texttt{toStringOneNumber}, która zakładam, 
że przyjmuje liczbę typu \texttt{\textunderscore \textunderscore int128}
i zwraca jej zapis jej reszty z dzielenia przez \texttt{10} w formnie zmiennej typu \texttt{string}.
Funkcja \texttt{toString} 
przyjmuje wartość \texttt{x} typu \texttt{\textunderscore \textunderscore int128}
Jeżeli \texttt{x==0} zwraca \texttt{"0"}. Jeżeli \texttt{x} jest ujemne zapamiętuje to w zmiennej
\texttt{minus} typu \texttt{bool} i następnie mnoży w miejscu \texttt{x} przez \texttt{-1}. 
Funkcja inicjalizuje zmienną wynikową \texttt{ans} na słowo puste, a następnie wykonuje pętlę
\begin{lstlisting}
    while(B != 0){
        if(B \% 2 == 1)ans *= multiplier;
        multiplier *= multiplier;
        B /=2;
    }
\end{lstlisting}
W przypadku gdy zmienna \texttt{minus} została ustawiona na \texttt{true} dodaję z przodu wyniku słowo \texttt{"\textminus"}.
Po wykonaniu omówionych operacji zwracam wynik. 



Omówiony w powyższej sekcji kod został napisany w języku $C++$ i znajduje się w 
pliku $field.h$ w dołączonym do pracy repozytorium.


\section{Szukanie liczby pierwszej}
W pracy Jin i Wu liczba pierwsza którą losujemy ma tylko jedno zastosowanie. Jest nim
wyznaczenie ciała w którym będziemy wykonywać nasze obliczenia. Losowanie jej odbywa się 
zpośród liczb na przedziale $[t+1,(n+t)^3]$. 

\begin{tcolorbox}
    %\begin{center}
        \textbf{Definicja:} $\pi(x)$ jest funkcją $\mathbb{N}\to\mathbb{N}$ przypisująca argumentowi $x$ liczbę 
        liczb pierwszych niewiększych niż $x$.
    %\end{center}    
\end{tcolorbox}

Twierdzeniem o liczbach pierwszych mówi, że: [ŹROOOOOOOOOOOOOOODŁOOOOOOOOOOOOO] 
$$\lim_{x \to \infty} \frac{\pi(x)}{\frac{x}{\ln(x)}}=1$$

Tak więc liczb pierwszych na przedziale $[t+1,(n+t)^3]$ jest $\Omega((n+t)^2)$. Celem naszego algorytmu jest sprawdzenie
czy dla danego multizbioru $\{x_1,x_2,...,x_n\} \equiv \mathbb{N}$ i liczby $t$ współczynnik przy $x^t$ w wielomianie 
$\prod_{i=1}^{t}(1+x^{s_j})$ jest dodatni. Ponieważ obliczenia wykonujemy w ciele $\mathbb{Z}_p$ jeżeli 
ów współczynnik byłby podzielny przez $p$ fałszywie uznalibyśmy go za zerowy. Jednak współczynnik ten będzie 
niewiększy od $2^n$, tak więc ma najwyżej $O(n)$ dzielników pierwszych, więc prawdopodobieństwo, że wylosowana
liczba dzieli ten współczynnik jest $O(\frac{1}{n+t})$

Różnica, między moją implementacją, a algorytmem opisanym przez Ce Jin i Hongxun Wu jest 
jednak taka, że nie używałem klasycznej wersji szybkiej transformaty Fouriera, opartej na
liczbach zespolonych, a teorio-liczbowej szybkiej transformaty Furiera wykonującej obliczenia
w ciele $\mathbb{Z}_p$. Ponieważ w tym ciele $\mathbb{Z}_p$ musi istnieć pierwiastek z 1 stopnia $r$ gdzie
$r$ jest postaci $2^k$, dla pewnej liczby naturalnej $k$, oraz $r$ jest większe niż 
stopnień wielomianu, który powstanie po pomnożeniu wielomianów, do których mnożenia używamy naszą 
transformatę. Mnożone wielomiany są stopnia co najwyżej $t$, tak więc
możemy przyjąć, że $2^k$ to najmniejsza potęga $2$ o całkowitym wykładniku większa niż 
$2t$. zaś $p$ ma postać $r2^k+1$. 
\begin{tcolorbox}
    \textbf{Twierdzenie:}Jeżeli w ciele $\mathbb{Z}_p$ istnieje pierwiastek stopnia $2^k$ z $1$ to $p$ ma postać $r2^k+1$, gdzie $r$ jest liczbą naturalną.
\end{tcolorbox}

Istotnie dla każdej liczby pierwszej $p$ istnieje jej pierwiastek pierwotny $a$,
którego stopień jest równy $p-1$, czyli w tym przypadku $r2^k$, zaś stopień $a^r$ jest równy $2^k$.


Intuicjyjnie można przypuszczać, że ponieważ rozmieszczenie liczb pierwszych jest trudne 
do opisania prostym wzorem, to podobny procent liczb postaci $r2^k+1$ gdzie $r$ jest liczbą
naturalną z przedziału $[t+1,(n+t)^3]$ jest pierwszy, co po prostu, procent liczb 
pierwszych na przedziale $[t+1,(n+t)^3]$. 

\begin{tcolorbox}
    \textbf{Definicja}: $\pi'(x,t)$ to funkcja $(\mathbb{N} \times \mathbb{N}) \to \mathbb{N}$, ktora dla argumentów $x$ i
    $t$ zwraca liczbę liczb pierwszych postaci $r2^k+1$, gdzie $2^k$ to najmniejsza potęga $2$ o wykładniku naturalnym
    większa niż $2t$, zaś $r$ jest liczbą naturalną mniejszą lub równą $x$
\end{tcolorbox}

Losowanie liczby pierwszej $p$ o rządanych właściwościach będzie polegać najpierw na wylosowaniu liczby naturalnej $r$
z przedziału $[t+1,(n+t)^3]$, a następnie wyliczeniu liczby $k2^r+1$, gdzie $2^k$ to najmniejsza potęga $2$ o wykładniku naturalnym większa niż $2t$ i sprawdzeniu, czy tak wyliczona liczba jest pierwsza.

Dla danych $t$ i $n$ tym sposobem możemy wylosować $\pi'((n+t)^3,x)-\pi'(t+1,x)$ różnych liczb pierwszych. Jeżeli 
więc $\pi'((n+t)^3,x)-\pi'(t+1,x)$ jest conajmniej $\Omega((n+t)^2)$ to szacowanie prawdopodobieństwa błędu analogiczne do szacowania 
Jin i Wu zachodzi. Wyliczenie dokładnej wartości $\pi'((n+t)^3,x)-\pi'(t+1,x)$ dla dużych $n$ i $t$ jest trudne 
dlatego oszacowałem je poprzez losowanie \texttt{1000000} liczb z przedziału $[1+t,(n+t)^3]$, pomnożeniu przez $2^k$, gdzie $2^k$ to najmniejsza potęga $2$ o wykładniku naturalnym
większa niż $2t$ i dodaniu $1$, oraz zliczeniu w zmiennej \texttt{ans} ile tak wylosowanych liczb jest pierwsze. 
$\frac{ans}{1000000}((n+t)^3-(1+t)+1) \approx \pi'(n,t)$.

Poniższa heat-mapa ukazuje tak przybliżoną wartość $\log_2(\frac{\pi'(n,t)}{(n+t)^2})$, w zależności od $\log_2(n)$ i $\log_2(t)$. Jak widać zdecydowanie liczba liczb pierwszych możliwych do uzyskania w omówiony sposób
jest $\Omega((n+t)^2)$.

W przeciwieństwie do większości pozostałego kodu, który przygotowałem, ze względu na jednoczesne łatwe wykonywanie wykresów, oraz szybkość działania eksperymentalne weryfikowanie
tej hipotezy przeprowadziłem w języku Julia. Kod użyty do testowania znajduje się  w pliku \texttt{PrimesExperiment.jl}.

Omówię teraz już samą implementację modułu szukającego liczby pierwszej. Wszystkie wymienione niżej funkcje zostały  zaimplementowane w języku C++ i znajdują się w pliku \texttt{FindPrimes.cpp}.

Moduł losujący składa się z kilku funkcji, z których zdecydowanie najprostszą jest \texttt{pow2} która dla przyjmowanej jako argument liczby \texttt{t}
zwraca najmniejszą potęgę \texttt{2} z wykładnikiem naturalnym, która jest większa niż \texttt{2t}. Polega ona na
mnożeniu przez \texttt{2} zmiennej \texttt{ans} której początkowa wartość jest równa \texttt{1} dopóki nie będzie 
większa niż \texttt{2*t}. Musimy wykonać co najwyżej kilkadziesiąt mnożeń, co jest na tyle szybkie, że nie ma 
potrzeby stosowania jakiś bardziej zaawansowanych algorytmów jak na przykład 
używających algorytmu szybkiego potęgowania.

Funkcja \texttt{randomLongLong} służy losowaniu liczby dodatniej 64-bitowej niewiększej niż podana jako argument 
liczba \texttt{mod}. Należąca do standardu C++ funkcja \texttt{rand} zwraca liczby 32-bitowe, co może nie być dla nas
wystarczające. Jeżeli \texttt{mod} mieści się w zakresie zmiennej typu \texttt{int} zwracam po prostu resztę z dzielenia przez \texttt{mod} wartości bezwzględnej
wyniku wywołania \texttt{rand()}. Jeżeli zaś jest większe niż $2^{31}-1$ \texttt{31} ostatnich bitów wyniku zapisujemy w zmiennej \texttt{candidate} jako resztę z dzielenia przez $2^{30}$ wartości bezwzględnej wyniku wywołania funkcji \texttt{rand()},
a następnie losuję bity wiodące jako wynik  operacji \texttt{std::abs(((long long)std::rand())\%((long long)mod/(long long)1073741824))} i zapisuję je w zmiennej \texttt{candidate2}. Może się zdażyć, że niektóre liczby mniejsze niż 
\texttt{mod} będą nieosiągalne, jednak jest z pewnością mniej niż $\frac{mod}{2}$, więc nie będzie to wpływać na 
asymptotyczną złożoność prawdopodobieństwa błędu. 

Podobną konstrukcję ma funkcja \texttt{random128} służąca do losowania dodatniej liczby typu
\texttt{\textunderscore \textunderscore int128} z przedziału $[0,1,...,mod]$, gdzie \texttt{mod}
to liczba typu \texttt{\textunderscore \textunderscore int128} podana jako argument. 
Rolę jaką w funkcji \texttt{randomLongLong} pełnią funkcja \texttt{rand} i liczby $2^{31}-1$i $2^{30}$ w kodzie \texttt{random128} pełnią odpowiednio
funkcja \texttt{randomLongLong} i liczby $2^{63}-1$ i $2^{62}$. Nie musimy też w funkcji również troszczyć
się o wyciąganie wartości bezwzględnej bo funkcja \texttt{randomLongLong} zwraca tylko

Sprawdzenie czy wylosowana liczba jest pierwsza wykonuję przy pomocy testu Millera-Rabina. 

Jego idea jest następująca: każdą liczbę pierwszą $p$ większą niż $2$ da się jednoznacznnie zapisać w 
postaci $d2^s+1$
gdzie $d$ jest nieparzyste, zaś $s$ naturalne. Dla dowolnej liczby
$a \in \{2,3,...,p-1\}$ z małego twierdzenia Fermata $p$ dzieli

$$ a^{p-1}-1=a^{d2^s}-1=(a^d)^{2^s}-1=((a^d)^{2^{s-1}}+1)((a^d)^{2^{s-1}}-1)=$$


$$((a^d)^{2^0}-1)((a^d)^{2^0}+1)((a^d)^{2^1}+1)((a^d)^{2^2}+1)...((a^d)^{2^{s-1}}+1) $$

Tak więc z wyboru $a$ wiemy, że któraś z liczb  $((a^d)^{2^0}+1),((a^d)^{2^1}+1),...,((a^d)^{2^{s-1}}+1)$
jest podzielna przez $p$, czyli dla pewnego $i \in \{0,1,2,...,s-1\}$ zachodzi 

$$(a^d)^{2 ^i} + 1\equiv 0 (\mod p)$$
czyli równoważnie 
$$(a^d)^{2 ^i} + 1\equiv n-1 (\mod p)$$

Test Millera-Rabina opiera się na następującej obserwacji:
\begin{tcolorbox}
\textbf{Obserwacja 1:}Dla większej niż $2$ liczby $n=2^sd+1$, gdzie $s$ jest naturalne, zaś $d$ naturalne nieparzyste i liczby $
a \in \{2,3,...,n-1\}$
jeżeli istnieje liczba $i \in \{0,1,2,..,s-1\}$ taka, że $(a^d)^{2^i} \equiv n-1 (\mod p)$ to $n$
prawdopodobnie jest pierwsza. Jeżeli takie $i$ nie istnieje $n$ liczbą pierwszą nie jest na pewno.
\end{tcolorbox}

prawdopodobieństwo błędu można minimalizować wybierając kilka liczb $a$. 
\begin{tcolorbox}
    \textbf{Fakt 1:}Jeżeli hipoteza Riemanna jest 
    prawdziwa wystarczające okazuje się sprawdzenie sprawdzenie $a$ z mniejsze lub równe 
    $\max(n-2,2\ln^2(n))$.[ŹROOOOOOOOOOOOOOODŁOOOOOOOOOOOOO]
    
\end{tcolorbox}

Moja implementacja testu Millera Rabina składa z dwóch funkcji. 

Pierwsza z nich to \texttt{MillerRabinOne} mająca za zadanie sprawdzić warunek z \textbf{Obserwacji 1} dla określonych
\texttt{n} i \texttt{a} zwracając \texttt{false} gdy wiemy, że liczba nie jest pierwsza.
Na początku rozważam przypadki gdy \texttt{n} i \texttt{a} nie spełniają początkowych założeń dla których
\textbf{Hipoteza 1} była formułowana przy pomocy następujących zapytań warunkowych:
\begin{lstlisting}
    \begin{}t  
    if(n == 2) return true;
    if(n < 2) return false;
    if(n \% 2==0) return false;
    if(a \% n == 1) return true;
    if(a > n) return true;
\end{lstlisting}

Jeżeli warunki początkowe są spełnione przy pomocy funkcji \texttt{field::setPstupid} ustawiam wartości pól statycznych klasy 
\texttt{field} tak, żeby symulowała
ona obliczenia w pierścieniu $\mathbb{Z}_n$. Oczywiście nie wszystkie działania zdefiniowane
w klasie \texttt{field} działają w pierścieniu $\mathbb{Z}_n$(na przykład dzielenie). Jednak
możemy to zignorować po prostu ograniczając się do działań dodawania, odejmowania, mnożenia i potęgowania
z wykładnikiem naturalnym.

Wyznaczam \texttt{d} i \texttt{s} zgodnie z oznaczeniami z \textbf{Hipotezy 1}.

Następnie tworzę obiekt \texttt{x} klasy \texttt{field} z wartością pola \texttt{value} równą \texttt{a}.
Podnoszę \texttt{x} do potęgi \texttt{d} przy pomocy operatora \verb!^!\texttt{=} zdefiniowanego w module definiującym klasę \texttt{field}.
Następnie \texttt{s}-krotnie podnoszę w miejscu \texttt{x} do kwadratu przed każdym kolejnym podniesieniem sprawdzając, czy \texttt{x==n-1} 
co oznaczałoby, że liczba \texttt{n} jest "raczej pierwsza" i zwracamy wtedy \texttt{true} lub \texttt{x==1} co oznaczałoby, że \texttt{x} po kolejnych podniesieniach do kwadratu
będzie równe \texttt{1}, więc \texttt{n} nie jest pierwsza i zwracamy \texttt{false}. Jeżeli żaden z tych warunków nigdy nie zaszedł 
po \texttt{s-1} podniesieniach do kwadratu zwracamy \texttt{false}.

Kolejną funkcją jest \texttt{MillerRabin}, która przyjmuje liczbę \texttt{n} typu \texttt{\texttt{\textunderscore \textunderscore int128}} i 
wywołuje funkcję \texttt{MillerRabinOne} z \texttt{n} jako pierwszym argumentem oraz drugim będącym kolejnymi liczbami ze zbioru
$\{2,3,...,\min(n-2, 2\ln^2(n))\}$ i jeżeli, któreś z tych wywołań zwróci \texttt{false} funkcja \texttt{MillerRabin} również 
zwraca \texttt{false}. Jeżeli jednak tak nie będzie to na podstawie \textbf{Faktu 1} zwracamy \texttt{true} oznaczające, że 
\texttt{n} z bardzo małym prawdopodobieństwem błędu jest pierwsza. 

Kolejną już ostatnią funkcją w tym module jest funkcja \texttt{find\_prime}. Przyjmuje ona jako argumenty liczbę \texttt{n} i 
\texttt{t}, zaś następnie zwraca liczbę pierwszą postaci $r2^k+1$, gdzie $r$ jest liczbą naturalną z przedziału $[t+1,(n+t)^3]$.
Najpierw przy pomocy wywołania \texttt{srand((unsigned int)time(NULL)) ;} zwiększam losowoś testu. 
Następnie przy pomocy funkcji \texttt{pow2} wyznaczam $2^k$. Następnie losuję potencjalne liczby \texttt{candidate} przy pomocy wywołania
funkcji \texttt{random128} na $(n+t)^3$ . Jeżeli wylosuję liczbę niewiększą niż \texttt{t} losuję jeszcze raz. Dla dużych \texttt{n} i \texttt{n}
prawdopodobieństwo tego jest znikome, a ewentualny koszt niewielki. Następnie mnożę
wylosowane $r$ z wyznaczonym $2^k$ i dodaję do wyniku $1$. Kolejnym krokiem jest dla tak uzyskanej liczby sprawdzenie, czy 
jest pierwsza przy pomocy funkcji \texttt{MillerRabin}.

Omówione w tej sekcji funkcje znajdują się w pliku \texttt{find\_prime.cpp} i są napisane w języku C++.



\section{Teorio-liczbowa szybka transformata Furiera}

Dyskretna transformata Furiera służy przyspieszeniu mnożenia wielomianów. 
Wielomiany możemy reprezentować jako ciąg współczynników lub jako ciąg wartości w ustalonych punktach 
, których liczba przekracza stopień reprezentowanego wielomianu. Pierwsza reprezentacja jest wygodna
do wyliczania wartości w dowolnym punkcie z dziedziny, jednak mnożenie wielomianów w takiej postaci
jest czasochłonne i ma czas kwadratowy od długości wielomianów. Druga reprezentacja pozwala na 
mnożenie wielomianów w czasie liniowym, jednak wyliczanie ich wartości w innych punktach jest
trudne. Dyskretna transformata Fouriera służy do przechodzenia między tymi postaciami w 
relatywnie szybkim czasie $O(\log(n)n)$, gdzie $n$ to stopień tych wielomianów.

Wybór punktów opiera się na prostej obserwacji, że jeżeli $A(x) = a_0x^0+a_1x^1+..+a_{2n-1}x^{2n-1}$
(jeżeli stopień $A$ ma stopień parzysty przyjmujemy po prostu, że $a_{2n-1}$ jest równe $0$). jest równe
sumie $A_0(x^2)$ gdzie $A_0(t)$ jest równe $\sum_{i=0}^{n-1}a_{2i}t^{i}$, oraz 
$A_1(x^2)x$, gdzie $A_1(t)$ jest równe $\sum_{i=0}^{n-1}a_{2n+1}t^{i}$. Tak więc jeżeli dwie
liczby $a$ i $b$ mają te same wartości swoich kwadratów możemy obliczyć tylko raz wartości $A_1$ i $A_2$
od tego kwadratu, a następnie w czasie stałym połączyć te wyniki, tak aby uzyskać wartości 
wielomianu $A$ w punktach $a$ i $b$. 


Klasyczna wersja dyskretnej transformaty Furiera oblicza wartości wielomianu $A$ dla
argumentów z ciała liczb zespolonych. Przyjmujemy następujące oznaczenie $A(x)=\sum_{i=0}^{\infty}a_ix^i$,
przy czym istnieje $I$ takie, że $\forall i>I: a_i = 0$. I oznaczając dalej 
 $A_0(x) = a_0x^0+a_2x^1+..+a_{2^m-2}x^{2^{m-1}-1} $
oraz $A_1(x) = a_1x^1+a_3x^1+..+a_{2^m-1}x^{2^{m-1}-1} $, przy czym $2^m$ jest najmniejszą
potęgą $2$ o wykadniku naturalnym, większym niż stopień wielomianu będącego wynikiem optymalizowanego 
przez nas mnożenia. 

Będę oznaczał $\omega_m$ jako $e^{\frac{i2\pi}{2^m}}$. 
Zaóważmy, że :
\begin{equation*}
  \systeme{
  A(\omega_m^1) = A_0(\omega_m^2)+A_1(\omega_m^2)\omega_m^1 = A_0(\omega_{m-1}^1)+A_1(\omega_{m-1}^1)\omega_m^1,
  A(\omega_m^2) = A_0(\omega_m^4)+A_1(\omega_m^4)\omega_m^2 = A_0(\omega_{m-1}^2)+A_1(\omega_{m-1}^2)\omega_m^2,
  A(\omega_m^3) = A_0(\omega_m^6)+A_1(\omega_m^6)\omega_m^3 = A_0(\omega_{m-1}^3)+A_1(\omega_{m-1}^3)\omega_m^3,
  \cdot \cdot \cdot ,
  A(\omega_m^{2^m}) = A_0(\omega_m^{2^{m+1}})+A_1(\omega_m^{2^{m+1}})\omega_m^{2^m} = A_0(\omega_{m-1}^{2^m})+A_1(\omega_{m-1}^{2^m})\omega_m^{2^m}}
\end{equation*}

Tak więc gdy mamy wektory $V_0=\Bigl( A_0(\omega_{m-1}^0),A_0(\omega_{m-1}^1),...,A_0(\omega_{m-1}^{2^{m-1}})\Bigr)$ oraz                          

$V_1=\Bigl( A_1(\omega_{m-1}^0),A_1(\omega_{m-1}^1),...,A_1(\omega_{m-1}^{2^{m-1}})\Bigr)$ możemy
w czasie liniowym wyliczyć wektor $V=\Bigl( A(\omega_{m}^0),A(\omega_{m}^1),...,A(\omega_{m}^{2^{m}})\Bigr)$, jednak wyliczenie wektorów 
$V_1$ i $V_2$ można znów rozbić na wyliczenie najpierw $2$ wektorów długości $2^{m-2}$ i 
następnie połączenie tych dwóch wektorów w czasie liniowym od ich długośći. Możemy tak rozbijać kolejne wektory, aż dojdziemy do wektorów długości
$1$ które przechowują wartości pewnych wielomianów $0$-stopnia będący po prostu jednym ze współczynników $A$.

Niech $O(t)$ będzie czasem policzenia wartości wielomianu $A$ w $n=2^m$ punktach przy pomocy dyskretnej transformaty Furiera. 
Furiera. Wiemy, że $O(t)=2O(\frac{t}{2})+O(n)$. Rozwiązaniem takiego równania jest $O(t)=O(n\log(n))$.
Okazuje się, że po wyliczeniu wartości wielomianu $AB$ w omawianych punktach możemy "wyciągnąć" z tych wartości w czasie
$O(n)$ wartości współczynników wielomianu $AB$, gdzie $n$ to ilość tych punktów, tak więc cały algorytm mnożenia odbywa się w czasie $O(\log(n)n)+O(n)=O(\log(n)n)$

Moja implementacja kożysta z dwóch klasycznych ulepszeń dyskretnej transformaty Furiera. 

Pierwsza to tak zwana "Teorio-liczbowa transformata Furiera". Polega ona na nie wykonywaniu obliczeń w ciele $C$ lecz
w $Z_p$, przy czym oczywiście musi w ciele tym istnieć pierwiastek z $1$ stopnia $2^m$, gdzie $2^m$ jest potęgą $2$ o wykładniku
naturalnym i większa niż stopień zwracanego wielomianu. Pierwiastek ten też podniesiony do jakiejkolwiek potęgi 
o wykładniku naturalnym dodatnim, mniejszym niż $2^m$ musi być różny od $1$. 
Tak naprawdę oznacza to po prostu, że $p$ ma postać $r2^m+1$, gdzie $r$ jest liczbą nauralną, o czym pisałem w poprzedniej 
sekcji.

Jako $A[i:j]$ będę oznaczał komórki tablicy ze współczynnikami wielomianu $A$ o indeksach większych lub równych niż $i$ oraz mniejszych lub równych $j$. 


Drugim ulepszeniem jest użycie tak zwanej szybkiej transformaty Furiera. Opiera się ona na obserwacji że rekurencyjna wersja dyskretnej transformaty Furiera 
polega najpierw na podzieleniu współczynników wielomianu na coraz mniejsze grupy, aż do zawierających tylko pojedynczy współczynnik, a następnie łączeniu pojedynczych współczynników
w pary zawieerające informację o $2$ współczynnikach, potem czwórki, ósemki itd aż do połączenia ich w jeden duży blok zawierający informacje o wszystkich współczynnikach, a rekurencyjne wywołania służą jedynie 
pogrupowaniu w jakiej kolejności będziemy wykonywać łączenia.(NIE WIEM JAK TO NAPISAC LEPIEJ????)


Możemy jednak to grupowanie wykonać bez kolejnych rekurencyjnych wywoływań transformaty po prostu ustawiając obok siebie współczynniki w takiej kolejności 
żeby współczynniki wielomianów stopnia $2^l-1$ które w wersji rekurencyjnej powstają na $m-l$ poziomie rekursji
były w spójnych segmentach $A[0:2^l-1]$,$A[2^l:2 \cdot 2^l-1]$,
$A[2 \cdot 2^l:2(2^l)-1]$,...,$[2^m-2^l,2^m-1]$. Przy czym współczynniki wielomianu powstałego na
$l+1$ poziomie rekursji, które w wielomianie powstałym na $l$-tym poziomie rekursji były przy 
potęgach o wykładniku parzystym zostają umieszczone w "pierwszej połowie" segmentu 
długości $2^l$ w którym były umieszczone na poprzednim etapie, zaś te, które były przy 
potęgach nieparzystych w drugim.

Zaóważmy, że pierwsze "rozbicie" wektora współczynników w wersji rekurencyjnej sprowadza się,
do wybrania osobno współczynników parzystych i nieparzystych. Tak więc nasze grupowanie
powinno w polach o indeksach $(0,1,2,...,2^{m-1}-1)$ umieścić liczby znajdujące sję najpierw 
w polach o indeksach parzystych, czyli mające ostatni bit równy \texttt{0}, z kolei 
w polach o indeksach $(2^{m-1},2^{m-1}+1,...,2^m-1)$ powinno umieścić liczby nieparzyste, czyli
mające ostatni bit równy \texttt{1}. 

Przyjmijmy, że pierwszy poziom rekursji rozbija wielomian
$A(x)=\sum_{i=0}^{2^m-1}a_ix^i$, na $A_0(x)=\sum_{i=0}^{2^{m-1}-1}a_{2i}x^{i}$ oraz
$A_1(x)=\sum_{i=0}^{2^{m-1}-1}a_{2i+1}x^{i}$. 

Gdy umieścimy współczynniki $A_0$ w $A[0,2^{m-1}-1]$, zaś $A_1$ w $A[2^{m-1},2^m-1]$ należy umieścić współczynniki przy potęgach parzystych w wielomianie $A_0$ w pierwszej połówce 
$A[0,2^{m-1}-1]$, nieparzyste zaś w drugiej. I analogicznie współczynniki przy potęgach 
parzystych w wielomianie $A_1$ w pierwszej połówce 
$A[2^{m-1},2^{m}-1]$, nieparzyste zaś w drugiej. To który współczynnik jest przy potędze 
parzystej czy nie w wielomianach $A_0$ i $A_1$ decyduje to czy początkowo był przy potędze
której wykładnik miał drugi bit równy $0$, czy $1$. Dalsze kontynuowanie tego rozumowania 
prowadzi do wniosku, że jeżeli odwrócimy $m-1$ ostatnich bitów liczby $i$ otrzymamy numer komórki
w której powinniśmy umieścić liczbę $a_i$. Istotnie załóżmy, że liczba powstała po odwróceniu $m$ ostatnich bitów liczby $i$ jest większa niż że liczba powstała po odwróceniu $m$ ostatnich bitów liczby $j$. Prosta indukcja pozwala stwierdzić, że współczynniki, których indeksy mają $k$ 
wspólnych ostatnich 
bitów zostają na $k$-tym poziomie rekursji umieszczone w tym samym wielomianie. 
To czy zostaną umieszczone w pierwszej czy drugiej połowie segmentu do którego były przypisane 
na $k$-tym poziomie rekursji determinuje to czy ich $k+1$ bit jest równy $0$ czy $1$.

To, że liczba powstała po odwróceniu $m$ ostatnich bitów liczby $i$ jest większa niż liczba powstała po odwróceniu $m$ ostatnich bitów liczby $j$ oznacza, że pewna liczba $k$(być może równa 
$0$)ostatnich bitów liczby $i$ jest taka sama jak liczby $j$ 
zaś $k+1$ ostatni bit liczby $i$ jest równy $1$, podczas gdy $k+1$ ostatni bit liczby $j$
jest równy $0$.
Tak więc do $k-tego$ poziomu rekursji $a_i$ i $a_j$ pozostają w tym samym wielomianie $B(x)=
\sum^{2^{m-k}-1}_{l=0} b_lx^l$, czyli w wersji iteracyjnej są w tym samym segmencie 
długości $2^{m-k}$ i na $k+1$ poziomie rekursji $a_j$ jest umieszczane w wielomianie
ze współczynnikami $b_0,b_2,...,b_{2^{m-k}-2}$, zaś $a_i$ w wielomianie ze współczynnikami
$b_1,b_3,...,b_{2^{m-k}-1}$, tak więc w wersji iteracyjnej $a_j$ zostaje w umieszczone w pierwszej połowie 
segmentu w którym są współczynniki $b_0, b_1,...,b_{2^{m-k}-1}$ zaś $a_i$ w drugiej, tak więc $a_i$ zostaje umieszczone w komórce o większym indeksie niż $a_j$. Ponieważ indeksy tabicy $A[0:2^m-1]$ 
wyznaczają wszystkie możliwe binarne liczby $m$-bitowe to istotnie jeżeli odwrócimy $m-1$ ostatnich bitów liczby $i$ otrzymamy numer komórki
w której powinniśmy umieścić liczbę $a_i$.


Przejdźmy do dokładnego opisu naszej implementacji. 

Najpierw definiuję funkcję \texttt{enoughGoodRoot} która przyjmuje liczbnę \texttt{k} typu 
\texttt{\textunderscore \textunderscore int128}
 i zwraca element typu \texttt{field}, który jest pierwiastkiem z \texttt{1} stopnia $2^k$. 
Jest to po prostu podniesienie elementyu \texttt{field::almostPrimitiveRoot} (będącego pierwiastkiem z $1$ stopnia $2^{field::degreeOfDegree}$) \texttt{field::almostPrimitiveRoot-k}
razy do kwad





Kolejnym etapem jest odpowiednie ustawienie wartości w komórkach wektora ze współczynnikami wielomianu(nazwę go \texttt{coefficients}, tak by można było wykonać na niej iteracyjną wersję szybkiej transformaty Furiera. 



Funkcja \texttt{reverseBits} służy znalezienia indeksu komórki, w której powinna być umieszczona 
liczba znajdująca się początkowo w komórce o indeksie \texttt{x} gdzie \texttt{x} jest pierwszym 
argumentem funkcji \texttt{reverseBits} typu
\texttt{long long}, zaś drugim argumentem jest liczba \texttt{k} 
typu \texttt{long long} wyznaczająca ile ostatnich bitów \texttt{x} należy odwrócić.

Na początku zamieniamy miejscami bity o indeksach parzystych \texttt{x} z nieparzystymi.
Robimy to najpierw tworząc dwie kopie zmiennej \texttt{x}. W jednej z nich zerujemy 
(and-ując ją z odpowiednią maską) bity o indeksach parzych, zaś w drugiej nieparzystych.
Pierwszą kopię przesówamy o $1$ bit w prawo, tak, że bity o indeksach parzystych stały się bitami
o indeksach nieparzystych, zaś drugą w lewo i wykonujemy na tak zmodyfikowanych kopiach operację
\texttt{or}-a bitowego. 

Analogiczną operację jak omówiona w poprzednim akapicie dla bloków składających się z $1$ 
liczby wykonujemy na blokach $2$-bitowych, $4$-bitowych, $8$-bitowych, $16$-bitowych,
i $32$-bitowych. 

Tak zmodyfikowana zmienna \texttt{x} jest odwróconą bitowo początkową wartością zmiennej 
\texttt{x}. Ponieważ chcieliśmy odwrócić jedynie jej ostatnie \texttt{k} bitów przed 
zwróceniem \texttt{x} jako wyniku przesówamy ją jeszcze o \texttt{64-k} bitów w prawo.






Gdy mamy już funkcję \texttt{reverseBits} możemy stworzyć funkcję \texttt{setToDo} transformującą wektor współczynników w "surowej" formie w wektor gotowy
do wyliczenia jego szybkiej transformaty Furiera. Jako argumenty dla tej funkcji otrzymujemy wektor elementów typu \texttt{field} o nazwie \texttt{coefficients} oraz
zmienną \texttt{long long} o nazwie \texttt{size} oznaczającą rządaną długość wektora wynikowego(jest to odpowiednio duża potęga dwójki o wykładniku naturalnym). Na początku dopełniamy wektor 
\texttt{coefficients} elementami równymi \texttt{field(0)} do rozmiaru \texttt{size}. 
W zmiennej \texttt{log} zapisujemy wartość logarytmu o podstawie $2$ na argumencie będącym długością
wektora \texttt{coefficients}
Następnie w pętli z iteratorem \texttt{i} przechodzącym po wszystkich liczbach naturalnych od \texttt{0} do 
\texttt{size-1} i jeżeli \texttt{reverseBits(i,log) > i}(warunek ten służy temu, by każdy element został zamieniony dokładnie raz, zauważmy też, że \texttt{reverseBits(reverseBits(i,log),log)=i}) zamieniam w tabeli pole o indeksie \texttt{i} z polem o indeksie 
\texttt{reverseBits(i,log)} miejscami. Funkcja nic nie zwraca, a jedynie modyfikuje wektor podany jako argument.

Przejdźmy do funkcji \texttt{DFT}, która otrzymując wektor współczynników wielomianu \texttt{coefficients}, 
liczbę \texttt{size} nie mniejszą niż długość tego wektora i będący potęgą $2$ o wykładniku naturalnym oraz obiekt \texttt{omegaM} typu \texttt{field},
który jest pierwiastkiem z \texttt{1} stopnia \texttt{size} zwraca wektor zawierający wartości tego wielomianu w punktach kolejno $omegaM^0$, $omegaM^1$,...,$omegaM^{size}$. 

Poniżej umieściłem kod tej procedutry.

\begin{lstlisting}
void inline DFT(std::vector<field>&coefficients, long long size, field omegaM){
    setToDo(coefficients,size);
    std::stack<field> omegasM;
    while(omegaM != 1){
        omegasM.push(omegaM);
        omegaM *= omegaM;
    }
    long long m = 1;
    while(!omegasM.empty()){   
        field currentOmegaM= omegasM.top();
        std::cout<<std::endl;
        omegasM.pop();
        field omega = 1;
        m*=2;
        std::cout<<m;
        for(long long j = 0; j<m/2;j+=1){
            for(long long k = j; k<size; k+=m){
                field t = omega * coefficients[k+m/2];
                field u = coefficients[k];
                coefficients[k] = u+t;
                coefficients[k+m/2] =  u - t;
            }
            omega *= currentOmegaM;
        }
    }
}
\end{lstlisting}




Na początku przygotowywujemy wektor do wykonania dalszej części obliczeń wywołując zdefiniowaną w poprzednim paragrafie funkcję \texttt{setToDo} na wektorze 
\texttt{coefficients} i dla liczby \texttt{size}. 

Tworzymy stos na który kładziemy wyniki kolejnych złożeń podniesiania do kwadratu liczby \texttt{omegaM} aż do \texttt{-1}.

Póki stos nie będzie pusty będę w każdej iteracji pętli \texttt{while} ściągał z niego kolejne wartości i przypisywał je do zmiennej \texttt{currentOmegaM}.

W dalszej części będę oznaczał jako $\omega_i$ pierwiastek z $1$ stopnia $2^i$, w $Z_p$.

Przyjmijmgy oznaczenie, że na początku \texttt{i}-tej iteracji wektor \texttt{coefficients} ma postać $\frac{size}{2^{i-1}}$ bloków postaci
$(f_{i,k}(\omega_i^0),f_{i,k}(\omega_i^1),...,f_{i,k}(\omega^{2^{i-1}-1}_i))$. gdzie $k$ to numer bloku. Chcemy
następujące po sobie pary bloków połączyć w jeden blok postaci $(f_{i+1,k}(\omega_{i+1}^0),f_{i+1,k}(\omega_{i+1}^1),...,
f_{i+1,k}(\omega^{2^{i}-1}_{i+1}))$, przy czym 
$f_{i+1,k}(\omega_{i+1}^j) = f_{i,2k}(\omega_{i}^\frac{j}{2})+f_{i,2k+1}(\omega_{i}^{\frac{j}{2}})\omega_{i+1}^j$. 

Kolejne iteracje pętli z iteratorem \texttt{j} odpowiadają kolejnym punktom w których wyliczamy wartości odpowiednich wielomianów, zaś pętla z iteratorem \texttt{k}
numery kolejnych rozpatrywanych wielomianów. Ponieważ $\omega_{i+1}^{2^i}=-1$ wyliczenie $f_{i+1,k}(\omega_{i+1}^{j})$ i $f_{i+1,k}(\omega_{i+1}^{j+2^i})$ 
wyliczam w jednej iteracji pętli.(jak lepiej to napisać?)


Przejdźmy do pełnej procedury mnożenia. 
\begin{lstlisting}
std::vector<field>multiplication(std::vector<field> A, std::vector<field> B){
    long long size = 1;
    while(size < (long long)((A.size() + B.size()) +2 )){
        size *= 2;
    }
    long long l = log(size);
    field omegaM = enoughGoodRoot(l);
    DFT(A,size,omegaM);  
    DFT(B,size,omegaM);
    for(long long i=0;i<size;i++){
        A[i] = B[i] * A[i]; 
    }
    omegaM = field(1)/omegaM;
    DFT(A,size,omegaM);
    for(long long i=0;i<size;i++){
        A[i] /= field(size);
    } 
    return A;
}
\end{lstlisting}
Wektorów nie przekazuję przez referencje bo będę je modyfikował.

Najpierw wiliczamy wartość \texttt{size} będącą liczbą punktów w których będziemy liczyć wartości wielomianu $AB$(jest to najmniejsza liczba będąca potgą 
$2$ o wykładniku naturalnym większa niż stopień $AB$). Kolejnym etapem jest znalezienie pierwiastka stopnia \texttt{size} z \texttt{1} w $\mathbb{Z}_p$, nie będącym w tym ciele jednocześnie pierwiastkiem 
z $1$ niższego naturalnego stopnia. Następnie wyliczamy przez wywołania $DFT$ na odpowiednich argumentach wektor $(A(omegaM^0),A(omegaM^1),...,A(omegaM^{size}))$ oraz
$(B(omegaM^0),B(omegaM^1),...,B(omegaM^{size}))$, następnie w kopii wektora $A$ zapisuję wartości $AB$ w kolejnych punktach będące wartościami odpowiednich 
mnożeń.

Okazuje się, że aby zmienić, wektor $(AB(omegaM^0),AB(omegaM^1),...,AB(omegaM^{size}))$ w wektor kolejnych współczynników $AB$ wystarczy wywołać na nim $DFT$ z drugim argumentem równym $size$ i trzecim 
będącym równym odwrotności $omegaM$ w $Z_p$, a następnie podzielić w $Z_p$ każdy jego element przez $size$. 























\section{Pochodna algebraiczna i jej własności}
Klasyczna analityczna definicja pochodnej jest to przekształcenie funkcji $f(x)$ w funkcję $f'(x)$ taką, że
$f'(x)=\lim_{h \to 0}\frac{f(x+h)-f(x)}{h}$. Definicja taka traci jednak cały sens jeżeli chcemy zmienić dziedzinę
funkcji $f$ i $f'$ z liczb rzeczywistych na dziedzinę gdzie nie możemy zmniejszać $h$ w taki sposób by było 
dowolnie małe lecz niezerowe. Przykładem takiej dziedziny jest ciało $Z_p$. 

Zaóważmy, że jeżeli funkcja $f$ $\mathbb{R}\to \mathbb{R}$ jest gładka w okolicy $0$, możemy ją utożsamić z jej szeregiem Taylora $\sum_{i=0}^{\infty}f_ix^i$. 

Weźmy liczbę pierwszą $p$. W dalszej części sekcji będę zakładał, że współczynniki szeregu Taylora 
rozważanej funkcji są postaci 
$\frac{a_i}{b_i}$, gdzie $a,b \in \mathbb{Z}$ i  $b$ jest niepodzielne przez $p$. Przy takim założeniu napis  
$\sum_{i=0}^{\infty}\frac{a_i}{b_i}x^i$ zachowuje algebraiczny sens również w ciele reszt $Z_p$, w którym 
liczbę całkowitą utożsamiamy z jej resztą z dzielenia przez $p$. W języku szeregów Taylora pochodną można 
zdefiniować jako przekształcenie funkcji $f(x)=\sum_{i=0}^\infty f_ix^i$ w funkcję 
$f '(x)=\sum_{i=0}^{\infty}(i+1)f_{i+1}x^i$. 

Udowodnię teraz, że tak zdefiniowana pochodna, dla funkcji $f(x)=\sum_{i=0}^{\infty}f_ix^i$ i
$g(x)=\sum_{i=0}^{\infty}g_ix^i$ zachowuje własności $f'+g'=(f+g)'$, $f'-g'=(f-g)'$, $(fg)'=f'g+fg'$ oraz 
$f(g)'=f'(g)g'$.

\begin{tcolorbox}
    \textbf{Lemat:} $f'+g'=(f+g)'$ oraz $f'-g'=(f-g)'$
    
    \textbf{Dowód:} $f'+g'=\sum_{1}^{\infty}if_ix^{i-1}+\sum_{1}^{\infty}ig_ix^{i-1}=
    \sum_{1}^{\infty}i(f_i+g_i)x^{i-1}=(f+g)'$ i analogicznie
    
    $f'-g'=\sum_{1}^{\infty}if_ix^{i-1}-\sum_{1}^{\infty}ig_ix^{i-1}=
    \sum_{1}^{\infty}(f_i-g_i)x^{i-1}=(f-g)'$ 
\end{tcolorbox}

\begin{tcolorbox}
    \textbf{Lemat:} $(fg)'=f'g+fg'$.
    
    \textbf{Dowód:} $(fg)'=((\sum_{i=0}^{\infty}f_ix^i)(\sum_{i=0}^{\infty}g_ix^i))'=
    (\sum_{i=0}^{\infty}(\sum_{j=0}^if_jg_{i-j})x^i)'=
    \sum_{i=1}^{\infty}(\sum_{j=0}^if_jg_{i-j})ix^{i-1}=
    \sum_{i=0}^{\infty}(\sum_{j=0}^if_jg_{i+1-j})(i+1)x^{i}$ z kolei

    $f'g+fg'=(\sum_{0}^{\infty}(i+1)f_{i+1}x^{i-1})(\sum_{i=0}^{\infty}g_ix^i)+(\sum_{0}^{\infty}(i+1)g_{i+1}x^{i})(\sum_{i=0}^{\infty}f_ix^i)=
    \sum_{i=0}^\infty(\sum_{j=0}^{i}f_{i+1-j}g_{j}(i+1-j))x^i+
    \sum_{i=0}^\infty(\sum_{j=0}^{i}g_{i+1-j}f_{j}(i+1-j))x^i=
    \sum_{i=0}^\infty(\sum_{j=0}^{i}g_{i+1-j}f_{j}(i+1-j+j))x^i=
    \sum_{i=0}^\infty(\sum_{j=0}^{i}g_{i+1-j}f_{j})(i+1)x^i=(fg)'
    $
\end{tcolorbox}

\begin{tcolorbox}
    \textbf{Lemat:} $(f(g))'=f'(g)g'$.
    
    \textbf{Dowód:} Na początek przyjmijmy, że $f(x)=x^k$, gdzie $k$ jest liczbą naturalną. Dla $k=0$ i $k=1$ teza jest 
    spełniona. Niech teza jest spełniona dla $f(x)=x^l$ dla każdego naturalnego $l$ mniejszego niż $k$.
    Wtedy
    $(g^k)'=((g^{k-1})g)'=(g^{k-1})g'+g(g^{k-1})'=(g^{k-1})g'+g((k-1)g^{k-2}g')=
    (g^{k-1})g'+((k-1)g^{k-1}g')=kg^{k-1}g'$, tak więc na mocy zasady indukcji dla dowolnego $k$ naturalnego jeżeli $f(x)=x^k$ to $(f(g))'=f'(g)g'$ dla dowolnego $g$ postaci $\sum_{1}^{\infty}g_ix^i$. W takim razie dla dowolnego $f$
    postaci $\sum_{1}^{\infty}g_ix^i$ zachodzi  $(f(g))'=(\sum_{i=0}^{\infty}f_ig^i)'=\sum_{i=0}^{\infty}f_i(g^i)'=
    \sum_{i=0}^{\infty}f_i ig^{i-1}g'=f'(g)g'$.

    
    
    
    
    
    
    
    %$f'(g)g'=(\sum_{i=0}^{\infty}f_{i+1}(i+1)g(x)^i)(\sum_{i=0}^{\infty}(i+1)g_{i+1}x^i)$
    
\end{tcolorbox}


\section{Implementacja algorytmu Jin i Wu}
W tej części pracy dla funkcji $F(x)$, której szereg Taylora ma postać $\sum_{i=0}^{\infty}f_ix^i$ jako $F_t$ będziemy
oznaczać $\sum_{i=0}^{t}f_ix^i$.

Zasadnicza implementacja algorytmyu Jin i Wu składa się z $4$ funkcji. 

Pierwszą z nich jaką omówię jest funkcja \texttt{B}, której zadaniem jest wyliczenie pierwszych 
$t+1$ współczynników szeregu Taylora $B(x)=\ln(\prod_{i=1}^n(1+x^{s_i}))$.

Funkcja \texttt{B} wyznacza wektor $t+1$ pierszych wyrazów szeregu Taylora w ciele $Z_p$ następującej 
funkcji:
$$
B(x)=\ln(\prod_{i=1}^n(1+x^{s_i}))=\sum_{i=1}^n \ln(1+x^{s_i})=\sum_{i=1}^n(\sum_{j=1}^\infty
\frac{(-1)^{j-1}}{j}x^{s_i j})
$$

Niech $a_k$ będzie oznaczać liczbę elementów $S$ równych $k$.

Przy tak przyjętych oznaczeniach 

$$
B_t(x)=\sum_{i=1}^n (\sum_{j=1}^{\lfloor \frac{t}{s_i} \rfloor}\frac{(-1)^{j-1}}{j}x^{s_ij})
=\sum_{k=1}^t (\sum_{j=1}^{\lfloor \frac{t}{k} \rfloor}\frac{a_k(-1)^{j-1}}{j}x^{jk})$$

Spójrzmy na implementację funkcji $B$.

\begin{lstlisting}
std::vector<field> B(std::vector<field> s, long long t){
    std::vector<field> a(t+1,field(0));
    std::vector<field> ans(t+1,field(0));
    int K;
    for(int i =0; i<s.size(); i++){
        if(s[i].getValue()<=t)a[s[i].getValue()] ++;
    }
    for(long long k = 1; k <= t; k++){
        for(long long j = 1; j <= t/k ; j++){
            field x = field(-1);
            ans[k*j] = ans[k*j] + a[k]*(x^(j-1))/field(j);
        }
    }
    return ans;
}
\end{lstlisting}

Na początku tworzę wektor \texttt{a} którego \texttt{k}-ty element oznacza zdefiniowane wcześniej $a_k$. 
Reszta kodu jest prostym przetłumaczeniem matematycznej notacji sumy na składnię zawierającą pętle, przy 
czym pole $ans[i]$ odpowiada współczynnikowi przy $x^i$ w rozwinięciu w szereg Taylora funkcji $B(x)$.

Kolejne dwie funkcjie to \texttt{compute} i \texttt{mainCompute}. 



Ich celem jest wyliczenie $(\exp(B(x)))_t=(\sum_{i=0}^\infty \frac{B(x)^i}{i!})_t$

Algorytm ten bazuje na spostrzeżeniu, że żeby wyliczyć $G_t(x)$,gdzie $G(x)=\exp(F(x))$ i $F(x) =\sum_{i=1}^\infty f_ix^i$,
jeżeli znamy $G(0)$ wystarczy znaleźć wartości $f_0,f_1,...,f_t$. Co więcej $G_t(F(x))=G_t(F_t(x))$

Niech rozwinięcie w szereg Taylora funkcji $G$ ma postać $\sum^{\infty}_{i=0}g_ix^i$

Zauważmy, że $G'(x)=(\exp(F(x)))'=\exp(F(x))F'(x)=G(x)F'(x)$, tak więc
$\sum_{i=0}^{\infty}(i+1)g_{i+1}x^{i}=(\sum_{i=0}^{\infty}g_ix^i)(\sum_{i=1}^{\infty}if_{i}x^{i-1})=
\sum_{i=0}^{\infty}(\sum_{j=1}^{i+1}jf_jg_{i+1-j})x^i$

Tak więc $(i+1)g_{i+1}=\sum_{j=1}^{i+1}jf_jg_{i+1-j}$, czyli $g_{i+1}=(i+1)^{-1}(\sum_{j=1}^{i+1}jf_jg_{i+1-j})$.
Zauważmy, że $G(0)=g_0$, tak więc mając wyliczone $f_0,f_1,...,f_t$ jesteśmy w stanie wyliczyć $g_1,g_2,...,g_t$, wyliczając je po kolei.

Weźmy liczbę pierwszą $p>t$.

$B(x)=\ln(\prod_{i=0}^n(1+x^{s_i}))=\sum_{i=1}^{n}(\sum_{k=1}^{\infty}\frac{(-1)^{k-1}x^{s_ik}}{k}):=\sum_{i=0}^{\infty}b_ix^i$, więc każda z liczb $b_0,b_1,...,b_n$ da się zapisać jako liczba wymierna postaci $\frac{x}{y}$, gdzie $x,y$ to
względnie pierwsze liczby naturalne i $y$ nie jest podzielne przez $p$. Jeżeli $x$ i $y$ utożsamimy z resztą dzielenia ich 
przez $p$ to wartość wyrażenia $\frac{x}{y}$ da się wyliczyć w ciele $Z_p$, a z tego.

Niech $A(x)=\exp(B(x))=\prod_{i=0}^n(1+x^{s_i}):=\sum_{i=0}^{\infty}a_ix^i$.

Wiemy, że $a_{i+1}=(i+1)^{-1}(\sum_{j=1}^{i+1}b_ja_{i+1-j})$, a także $a_0=\prod_{i=0}^n(1+0^{s_i})=1$, więc każda
z liczb $a_i$, gdzie $i \in \{0,1,2,..,t \}$ da się przedstawić w postaci $\frac{X_i}{Y_i}$, gdzie $X_i,Y_i$ to
względnie pierwsze liczby naturalne i $Y_i$ nie jest podzielne przez $p$. Jeżeli $X_i$ i $Y_i$ utożsamimy z resztą dzielenia ich 
przez $p$ to wartość wyrażenia $\frac{X_i}{Y_i}$ da się wyliczyć w ciele $Z_p$ i wartość liczby $a_i$ dla naturalnego $i$
nie większego niż $t$ jest równa $0$ wtedy i tylko wtedy gdy dla jej postaci $\frac{X_i}{Y_i}$ $p$ dzieli $X_i$. Zauważmy,
że $A(x)$ jest iloczynem wielomianów o całkowitych współczynnikach, więc też jest wielomianem o całkowitych współczynnikach,
więc $a_i$ jest wartością całkowitą, która w ciele $Z_p$ jest utożsamiona z $0$ wtedy i tylko wtedy, gdy jest podzielna przez $p$.

Na mocy powyższych rozważań, możemy dalsze obliczenia wykonywać w ciele $Z_p$, chyba, że explicite zaznaczę inaczej.


Procedura \texttt{mainCompute} jako argument przyjmuje wektor \texttt{f} $t+1$ pierwszych współczynników rozwinięcia w 
szereg Taylora funkcji $B$. Na początku inicjuje wynikowy $t+1$-elementowy wektor \texttt{g} ustawiając wszystkie
jego komórki poza \texttt{g[0]} na $0$, z kolei \texttt{g[0]} ustawiamy na $1$. Następnie uruchamiamy funkcję 
\texttt{compute} na argumentach \texttt{0}, \texttt{t}, \texttt{g} i \texttt{f}.

Funkcja \texttt{compute} zapisuje do wektora podanego jako trzeci argument(oczywiście przez referencję, żeby 
można było go modyfikować) wartości kolejnych współczynników szeregu Taylora funkcji 
$g(x)=\exp(\sum_{i=0}^tf_ix^i)$, gdzie $t+1$ to długość wektora podanego jako czwarty argument, zaś
$f_i$ to wartość $i$-tej komórki wektora podanego jako czwarty argument. W dalszej części jako $f_i$ będę 
oznaczał wartość $i$-tej komórki wektora podanego jako czwarty argument, zaś jako $g_i$ będę 
$i$-tej komórki wektora podanego jako trzeci argument. Jako $t$ będę oznaczał \texttt{f.size()-1}

Idea funkcji \texttt{compute} bazuje na tym, że aby wyliczyć $g_i=(i+1)^{-1}(\sum_{j=1}^{i+1}jf_jg_{i+1-j})$ dla wszystkich $i \in \{1,2,...,t\}$ należy
wykonać $O(n^2)$ dodawań składników postaci $(i+1)^{-1}jf_jg_{i+1-j}$ do odpowiednich komórek. Nie każde dodawanie można wykonać w dowolnym
momencie, ponieważ wartości komórek wektora \texttt{g} ulegają zmianie. Dodawanie uznamy za dozwolone, jeżeli $g_{i+1-j}$ obecne w 
dodawanym składniku $(i+1)^{-1}f_jg_{i+1-j}$ nie ulega zmianie. 

Zapisana w pseudokodzie funkcja \texttt{compute} ma postać:

\begin{lstlisting}
procedure compute(l,r,g,f)
    if l < r
        m <- floor((l+r)/2) #obliczenia w tej linii wykonujemy w liczbach naturalnych
        compute(l,m,g,f)
        for i <- m+1, m+2,...,r
            for j<-l,l+1,..,m
                g[i] <- g[i]+(i-j)f[i-j]g[j]/i
            end for
        end for
        compute(m+1,r,g,f)
    end if
end procedure
\end{lstlisting}

Po wykonaniu \texttt{compute(l,r,f,g)} chcemy, żeby wartości $g_l,g_{l+1},...,g_{r}$ były już ustawione na wartości docelowe, zaś
przed wykonaniem \texttt{compute(l,r,f,g)} chcemy, żeby wszystkie składniki postaci $(i+1)^{-1}f_jg_{i+1-j}$ gdzie $i+1-j<l$ zostały
już dodane do odpowiednich komórek \texttt{g} o indeksach należących do $\{0,1,...,r\}$. Chcemy też, żeby każda operacja dodawania
była dozwolona. 

Jeżeli długość \texttt{g} jest równa $0$ to rządania napisane w poprzednim akapicie są spełnione. 
Załóżmy indukcyjnie, że są spełnione dla \texttt{g} długości $0,1,2,..,t-1$. Niech długość \texttt{g} jest równa $t$.

Wywołanie \texttt{compute(0,t,f,g)} sprowadza się do wywołania \texttt{compute(0,m,f,g)}, gdzie \texttt{m} to wynik wykonanej w
liczbach całkowitych działania $\lfloor \frac{t}{2}\rfloor$. Po jej wykonaniu z tezy indukcyjnej $g_0,g_2,...,g_m$ mają docelowe
wartości. Następnie przechodzimy do wykonania pętli
\begin{lstlisting}
for i <- m+1, m+2,...,r
    for j<-l,l+1,..,m
        g[i] <- g[i]+(i-j)f[i-j]g[j]/i
    end for
end for
\end{lstlisting}

Ponieważ indeks $j$ jest nie większy niż $m$ wszystkie dodawania są dozwolone. 
Po wykonaniu tej pętli zostają już wykonane wszystkie
potrzebne dodania wyrazów postaci $g[i]+(i-j)f[i-j]g[j]/i$, gdzie $j$ jest niewiększe niż $m$. 
Następnie wykonujemy $compute(m+1,r,g,f)$, co przypomina wywołanie $compute(0,r-m-1,f,g)$ z tą modyfikacją, że w momencie 
gdy wykonalibyśmy linię \texttt{g[i] <- g[i]+(i-j)f[i-j]g[j]/i} każde wystąpienie
zmiennych \texttt{i} i \texttt{j} zastępujemy odpowiednio zmiennymi \texttt{i+m+1} i \texttt{j+m+1}. Ponieważ zgodnie z tezą
indukcyjną po wykonaniu \texttt{$compute(0,r-m-1,f,g)$}
$g_i$ zostaje zwiększone o $(i+1)^{-1}(\sum_{j=1}^{i+1}jf_jg_{i+1-j})$, dla $i \in \{1,2,...,r-m-1\}$ 
to po wykonaniu \texttt{compute(m+1,r,f,g)} $g_{i+m+1}$  zostaje zwiększone o $(m+1+i+1)^{-1}(\sum_{j=1}^{i+1}jf_{m+1+j}g_{m+1+i+1-j})$, 
dla $i \in \{1,2,...,r-m-1\}$, więc po wykonaniu $compute(m+1,r,g,f)$ $g_i=(i+1)^{-1}(\sum_{j=1}^{i+1}jf_jg_{i+1-j})$ dla każdego
$i \in \{0,1,2,3,...,t\}$. 

Nasza implementacja jednak kożysta z jednego usprawnienia. Zauważmy, że iloczyn wielomianów $F(x)=\sum_{k=0}^{r-l}kf_kx^k$ i 
$G(x)=\sum_{j=0}^{m-l}g_{j+l}x^j$, ma postać 
$\sum_{i=0}^{r+m-2l}(\sum_{k=0}^{r-l}kf_kg_{i-k+l})x^i$, tak więc w tym iloczynie współczynnik przy potędze $x^{i-l}$ podzielony
przez $i$ jest równy liczbie o którą zostaje zwiększone $g_i$ po wykonaniu pętli
\begin{lstlisting}
for i <- m+1, m+2,...,r
            for j<-l,l+1,..,m
                g[i] <- g[i]+(i-j)f[i-j]g[j]/i
            end for
end for
\end{lstlisting}

Wykonanie pętli naiwnie zajmuje czas $O(t^2)$, zaś wykożystanie szybkiej teorio-liczbowej transformaty Fouriera do mnożenia wielomianów,
w mojej implementacji przyspieszyć wykonanie tej pętli do $O(t\ln(t))$.
Niech $T(t)$ oznacza czas wykonania \texttt{compute}, gdzie różnica między drugim i pierwszym argumentem wynosi $t+1$.
Ponieważ $T(t)=2T(\frac{t}{2})+O(t\ln(t))$, to $T(t)=O(t \ln^2(t))$.

Kolejną ostatnią już implementowaną przezemnie funkcją jest \texttt{JinWu}. Przyjmuje ona wektor \texttt{s} elementów zbioru 
$S$, oraz
liczbę $t$. Najpierw losuje liczbę pierwszą $p$ która jest wynikiem wykonania funkcji \texttt{find\_prime(s.size(),t)}. Następnie
wywołując \texttt{field::setP(p)} ustawiam zmienne statyczne klasy \texttt{field} tak, żeby obliczenia w niej wykonywane 
odpowiadały wykonaniu ich w klasie $Z_p$. Potem do wektora $Bans$ przy pomocy wykonania funkcji $B(s,t)$ zapisuję
$t+1$ pierwszych współczynników(w $Z_p$) szeregu Taylora funkcji $B(x)=\ln(\prod_{i=0}^{n-1}(1+x^{s_i}))$, gdzie jako $n$ oznaczam długość
wektora \texttt{s}, zaś jako $s_i$ oznaczam jego $i$-tą komórkę w $Z_p$(będę to oznaczenie stosował również w dalszej części pracy). 
Następnie do wektora \texttt{computeAns} zapisuję $t+1$ pierwszych współczynników(w $Z_p$) szeregu Taylora funkcji 
$A(x)=\prod_{i=0}^{n-1}(1+x^{s_i})$ jako wynik procedury \texttt{mainCompute(t,Bans)}. 
Zauważmy, że ponieważ $A(x)$ jest wielomianem współczynnik
przy potędze $x^i$ w jego rozwinięciu w szereg Taylora jest po prostu współczynnikiem przy potędze $x^i$ w wielomianie $A(x)$, zaś współczynnik przy potędze $x^i$ w $A(x)$ oznacza ilość sposobów wybrania ciągu indeksów naturalnych $-1<i_1<i_2<...<i_m<n$, 
takich, że
$x^{s_{i_1}+s_{i_2}+...+s_{i_m}}=x^i$, czyli ilość podzbiorów $S$ dających sumę $i$. Ponieważ liczby te zapisujemy jako elementy 
$Z_p$ możliwe, \texttt{computeAns[t]} jest równe $0$ mimo, że istnieją podzbiory $S$ o sumie $t$ jednak ich liczba jest podzielna
przez $p$. Jednak liczba tych podzbiorów napewno jest niewiększa niż $2^n$, więc ma co najwyżej $n$ różnych czynników pierwszych.
Eksperymenty w sekcij poświęconej losowaniu liczby pierwszej zostało wykazane, że liczba możliwych do wylosowania wartości $p$
jest $O(\frac{(n+t)^3}{\log(n+t)})=O((n+t)^2)$, więc prawdopodobieństwo, że $p$ dzieli niezerową liczbę podzbiorów $S$ o sumie $t$
jest $O(\frac{n}{(n+t)^2})=O(\frac{1}{n+t})$.



\end{document}
