\documentclass{article}
\usepackage[T1]{fontenc}
% Language setting
% Replace `english' with e.g. `spanish' to change the document language
\usepackage[polish]{babel}

% Set page size and margins
% Replace `letterpaper' with `a4paper' for UK/EU standard size
\usepackage[letterpaper,top=2cm,bottom=2cm,left=3cm,right=3cm,marginparwidth=1.75cm]{geometry}

% Useful packages
\usepackage{amsmath}
\usepackage{graphicx}
\usepackage[colorlinks=true, allcolors=blue]{hyperref}

\title{Algorytm znajdowania sumy podzbioru}
\author{You}

\begin{document}
\maketitle

\begin{abstract}
Your abstract.
\end{abstract}

\section{Wstęp}
Problem sprawdzania czy z $n-$elementowego muiltizbioru $S$ liczb naturalnych jesteśmy w stanie wybrać podzbiór, którego suma elementów jest równa zadanej liczbie $t$ jest jednym z klasycznych problemów algorytmicznych. W nieniejszej pracy zaprezentuję niedeterministyczny algorytm opracowany przez Ce Jin i Hongxun Wu, który kożystając ze sprytnych obserwacji na polu analizy matematycznej i algebry jest w stanie podać wynik w czasie $O(n+tlog^2(t))$, co jest czasem znacznie szybszym niż klasyczny algorytm oparty na programowanie dynamiczne. 

\section{Algorytm klasyczny}

Na wejściu otrzymujemy multizbiór liczb naturalnych $S=\{s_1,s_2,...,s_n\}$, oraz liczbę naturalną $t$. Chcemy odpowiedzieć na pytanie czy jest możliwe wybranie $S' \subsetq S$ taki, że suma jego elementów jest równa $t$, przy czym wskazanie tego podzbioru nie jest konieczne, a wystarczy nam jedynie odpowiedź Tak lub Nie. 

Klasyczny algorytym polega na stworzeniu t-elementowej tablicy przechowującej wartości $0$ i $1$(dlatego najoptymalniej jest użyć do tego bitsetu). 

aaaaaaaaaaaaaaaaaaaaaaaaaaaaaaaaaaaaaaaaaa




\end{document}
